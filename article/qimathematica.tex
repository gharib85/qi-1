\documentclass[preprint,12pt]{elsarticle}
\usepackage{graphicx}

\usepackage{amssymb}

\journal{Computer Physics Communications}

\begin{document}
\begin{frontmatter}
\title{QI quantum information package for Mathematica}
%
\author[iitis]{Piotr Gawron}
%
\author[iitis]{Jaros{\l}aw Adam Miszczak}
%
\author[iitis]{Zbigniew Pucha{\l}a}
%
\address[iitis]{Institute of Theoretical and Applied Informatics, Polish Academy
of Sciences, Ba{\l}tycka 5, 44-100 Gliwice, Poland}

\begin{abstract}
%% Text of abstract

\end{abstract}

\begin{keyword}
quantum information theory \sep symbolic computation

%% MSC codes here, in the form: \MSC code \sep code
%% or \MSC[2008] code \sep code (2000 is the default)

\end{keyword}

\end{frontmatter}

%%
%% Start line numbering here if you want
%%
% \linenumbers

%% main text
\section{Introduction}
\label{sec:introduction}
\begin{itemize}
\item Motivation
\item Simmilar work \cite{Tabakin_Julia-Diaz_2011}, \cite{Hudson_2008}
\end{itemize}

\section{Quantum information theory}
\label{sec:qitheory}
\begin{itemize}
\item State vectors and density matrices
\item Unitary evolution and generalized evolution
\end{itemize}
%% The Appendices part is started with the command \appendix;
%% appendix sections are then done as normal sections
%% \appendix

%% \section{}
%% \label{}

\bibliographystyle{phcpc}
\bibliography{qimathematica}


\end{document}
