% v. 0.01 - 11/01/2012 - initial version 
% v. 0.02 - 11/01/2012 - 
% v. 0.03 - 25/01/2012 - Gawron - 
\documentclass[11pt,a4paper]{article}
\date{25/01/2012 (v. 0.03)}

\usepackage{fullpage}
\usepackage{amsmath,amsfonts}
%%%%%%%%%%%%%%%%%%%%%%%%%%%%%%%%%%%%%%%%%%%%%%%%%%%%%%%%%%%%%%%%%%%%%%%%%%%%%%%%
\newcommand{\qi}{\texttt{QI}}
\newcommand{\Mathematica}{\emph{Mathematica}}
\newcommand{\res}{\mathrm{res}}
\newcommand{\vek}{\mathrm{vek}}
\newcommand{\unres}{\mathrm{unres}}
\newcommand{\unvek}{\mathrm{unvec}}
\newcommand{\reshuffle}[1]{{#1}^{\mathcal{R}}}
\newcommand{\ket}[1]{\ensuremath{|#1\rangle}}
\newcommand{\bra}[1]{\ensuremath{\langle#1|}}
\newcommand{\ketbra}[2]{\ensuremath{\ket{#1}\bra{#2}}}
\newcommand{\scalar}[2]{\ensuremath{\langle#1\ket{#2}}}
\newcommand{\Hilb}[1]{\mathcal{H}_{#1}}
\newcommand{\Lin}{\mathrm{L}}
\newcommand{\tr}{\mathrm{tr}}
\newcommand{\1}{{\bf 1}}
\newcommand{\ii}{\mathrm{i}}
\newcommand{\fname}[1]{\text{\texttt{#1}}}



%%%%%%%%%%%%%%%%%%%%%%%%%%%%%%%%%%%%%%%%%%%%%%%%%%%%%%%%%%%%%%%%%%%%%%%%%%%%%%%%


%%%%%%%%%%%%%%%%%%%%%%%%%%%%%%%%%%%%%%%%%%%%%%%%%%%%%%%%%%%%%%%%%%%%%%%%%%%%%%%%
\newtheorem{lemma}{Lemma}
\newtheorem{theorem}{Theorem}
\newtheorem{fact}{Fact}
%%%%%%%%%%%%%%%%%%%%%%%%%%%%%%%%%%%%%%%%%%%%%%%%%%%%%%%%%%%%%%%%%%%%%%%%%%%%%%%%

%%%%%%%%%%%%%%%%%%%%%%%%%%%%%%%%%%%%%%%%%%%%%%%%%%%%%%%%%%%%%%%%%%%%%%%%%%%%%%%%
\title{\qi: A package for the analysis of quantum states and operations in
\Mathematica}

\author{Jarosław Adam Miszczak \and Zbigniew Pucha\l{}a \and Piotr Gawron}
%%%%%%%%%%%%%%%%%%%%%%%%%%%%%%%%%%%%%%%%%%%%%%%%%%%%%%%%%%%%%%%%%%%%%%%%%%%%%%%%

\begin{document}

\maketitle

%%%%%%%%%%%%%%%%%%%%%%%%%%%%%%%%%%%%%%%%%%%%%%%%%%%%%%%%%%%%%%%%%%%%%%%%%%%%%%%%
\section{Introduction}
%%%%%%%%%%%%%%%%%%%%%%%%%%%%%%%%%%%%%%%%%%%%%%%%%%%%%%%%%%%%%%%%%%%%%%%%%%%%%%%%

%%%%%%%%%%%%%%%%%%%%%%%%%%%%%%%%%%%%%%%%%%%%%%%%%%%%%%%%%%%%%%%%%%%%%%%%%%%%%%%%
\subsection{Related work}
%%%%%%%%%%%%%%%%%%%%%%%%%%%%%%%%%%%%%%%%%%%%%%%%%%%%%%%%%%%%%%%%%%%%%%%%%%%%%%%%

%%%%%%%%%%%%%%%%%%%%%%%%%%%%%%%%%%%%%%%%%%%%%%%%%%%%%%%%%%%%%%%%%%%%%%%%%%%%%%%%
\subsection{Preliminaries}
%%%%%%%%%%%%%%%%%%%%%%%%%%%%%%%%%%%%%%%%%%%%%%%%%%%%%%%%%%%%%%%%%%%%%%%%%%%%%%%%

%%%%%%%%%%%%%%%%%%%%%%%%%%%%%%%%%%%%%%%%%%%%%%%%%%%%%%%%%%%%%%%%%%%%%%%%%%%%%%%%
\section{Description of the package}
%%%%%%%%%%%%%%%%%%%%%%%%%%%%%%%%%%%%%%%%%%%%%%%%%%%%%%%%%%%%%%%%%%%%%%%%%%%%%%%%

%%%%%%%%%%%%%%%%%%%%%%%%%%%%%%%%%%%%%%%%%%%%%%%%%%%%%%%%%%%%%%%%%%%%%%%%%%%%%%%%
\subsection{Algebraic tools}
%%%%%%%%%%%%%%%%%%%%%%%%%%%%%%%%%%%%%%%%%%%%%%%%%%%%%%%%%%%%%%%%%%%%%%%%%%%%%%%%

%%%%%%%%%%%%%%%%%%%%%%%%%%%%%%%%%%%%%%%%%%%%%%%%%%%%%%%%%%%%%%%%%%%%%%%%%%%%%%%%
\subsubsection{Symbolic matrices}
%%%%%%%%%%%%%%%%%%%%%%%%%%%%%%%%%%%%%%%%%%%%%%%%%%%%%%%%%%%%%%%%%%%%%%%%%%%%%%%%
The power of \Mathematica{} lies in its ability to operate on symbolic objects.
To facilitate work with \qi{} the package there are several functions that 
return various types of symbolic matrices.

Function $\fname{SymbolicMatrix}[a,d]$ returns $d$-dimensional square matrix 
$\{a_{ij}\}$, the same function called with three parameters 
$\fname{SymbolicMatrix}[a,d_1,d_2]$ returns $d_1 \times d_2$-dimensional 
matrix $\{a_{ij}\}$. Accordingly one can call function 
$\fname{SymbolicVector}[a,d]$ to obtain $d$-dimensional vector $(a_i)$.
A~$d$-dimensional Hermitian matrix can be obtained using function 
$\fname{SymbolicHermitianMatrix}[a,d]$. 

For example a call to $\fname{SymbolicHermitianMatrix}[a,3]$ gives the 
following result
\begin{equation}
\left(
\begin{array}{ccc}
 a_{1,1} & \left(a_{2,1}\right){}^* & \left(a_{3,1}\right){}^* \\
 a_{2,1} & a_{2,2} & \left(a_{3,2}\right){}^* \\
 a_{3,1} & a_{3,2} & a_{3,3} \\
\end{array}
\right).
\end{equation}

%And a~bistochastic matrix using function
%$\fname{SymbolicBistochasticMatrix}[a, d]$.

%%%%%%%%%%%%%%%%%%%%%%%%%%%%%%%%%%%%%%%%%%%%%%%%%%%%%%%%%%%%%%%%%%%%%%%%%%%%%%%%
\subsubsection{Kronecker product and friends}
%%%%%%%%%%%%%%%%%%%%%%%%%%%%%%%%%%%%%%%%%%%%%%%%%%%%%%%%%%%%%%%%%%%%%%%%%%%%%%%%

%%%%%%%%%%%%%%%%%%%%%%%%%%%%%%%%%%%%%%%%%%%%%%%%%%%%%%%%%%%%%%%%%%%%%%%%%%%%%%%%
\subsubsection{Matrix reorderings}
%%%%%%%%%%%%%%%%%%%%%%%%%%%%%%%%%%%%%%%%%%%%%%%%%%%%%%%%%%%%%%%%%%%%%%%%%%%%%%%%

%%%%%%%%%%%%%%%%%%%%%%%%%%%%%%%%%%%%%%%%%%%%%%%%%%%%%%%%%%%%%%%%%%%%%%%%%%%%%%%%
\subsubsection{Schmidt decomposition (ZP)}
%%%%%%%%%%%%%%%%%%%%%%%%%%%%%%%%%%%%%%%%%%%%%%%%%%%%%%%%%%%%%%%%%%%%%%%%%%%%%%%%
In this section we present Schmidt decomposition for vectors and operators
on bipartite Hilbert space.

\paragraph{Vector Schmidt decomposition}
\begin{theorem} \label{th:Schmidt-Decomposition}
For a given vector  $\ket{\psi} \in \Hilb{K}\otimes \Hilb{M}$,
let $\Psi \in \Lin(\Hilb{M},\Hilb{K})$, such that $\res(\Psi) = \ket{\psi}$.
There exists a decomposition
\begin{equation}
 \ket{\psi} = 
 \sum \sqrt{\lambda_i} \ket{U(i)} \otimes \overline{\ket{V(i)}},
\end{equation}
where 
\begin{itemize}
 \item $\sqrt{\lambda_i}$ are singular values of matrix $\Psi$,
 \item $\ket{U(i)}$ is eigenvector of
$\Psi\Psi^{\dagger}$
 with the eigenvalue $\lambda_i$,
 \item $\ket{V(i)}$ is eigenvector of 
$\Psi^{\dagger} \Psi$
 with the eigenvalue $\lambda_i$.
\end{itemize}
\end{theorem}

To perform Schmidt decomposition on a vector $\psi \in \Hilb{K}\otimes
\Hilb{M}$ in package \qi{} one can call function
\begin{equation}
\text{\texttt{SchmidtDecomposition}}[ \psi, \{K,M\}]
\end{equation} 
in the case, when $K=M$, the dimension specification can be omitted.
See the function list in the appendix for specification of the function
\texttt{SchmidtDecomposition}.
%Example of performing Schmidt decomposition in the package \qi. 

...

\paragraph{Operator Schmidt decomposition}
\begin{theorem} \label{th:Operator-Schmidt-Decomposition}
Let $A \in \Lin(\Hilb{m_1}\otimes \Hilb{m_2},\Hilb{n_1}\otimes \Hilb{n_2})$,
then there exists decomposition
\begin{equation}
 A = \sum \sqrt{\lambda_i} U(i) \otimes \overline{V(i)},
\end{equation}
where 
\begin{itemize}
 \item $\sqrt{\lambda_i}$ are singular values of matrix $\reshuffle{A}$,
 \item $U(i) \in \Lin(\Hilb{m_1}, \Hilb{n_1})$, such that $\res(U(i))$ is eigenvector of $\reshuffle{A} (\reshuffle{A})^{\dagger}$
 with the eigenvalue $\lambda_i$,
\item $V(i) \in \Lin(\Hilb{m_2}, \Hilb{n_2})$, such that $\res(V(i))$ is eigenvector of $(\reshuffle{A})^{\dagger} \reshuffle{A}$
 with the eigenvalue $\lambda_i$.
\end{itemize}
\end{theorem}

To perform operator Schmidt decomposition on an operator $A \in \Lin(\Hilb{m_1}\otimes \Hilb{m_2},\Hilb{n_1}\otimes \Hilb{n_2})$ 
in the package \qi{} one can call function
\begin{equation}
\fname{SchmidtDecomposition}[ A, \{... , ... \}]
\end{equation} 
in the case, when ...... $K=M$, the dimension specification can be omitted.
See the function list in the appendix for specification of the function
\texttt{SchmidtDecomposition}.

%Let $A \in M_{n_1 n_2,m_1 m_2}$. 
%If we write singular value decomposition (see Theorem \ref{th:SVD})for reshuffled matrix $A$, we get
%\begin{equation}
% A^{R} = \sum \sigma_i \ketbra{U(i)}{V(i)}.
%\end{equation}
%Because $(A^{R})^{R} = A$ we get
%\begin{equation}
% A = (A^{R})^{R} = \sum \sigma_i (\ketbra{U(i)}{V(i)})^R.
%\end{equation}
%Now we use our previous result concerning reshuffling of rank 1 partial isometries 
%Eqn. (\ref{Eqn:Reshuffling-for-rank1-partial-isometry}) and obtain 
%\begin{equation}
% A = \sum \sigma_i (\ketbra{U(i)}{V(i)})^R =
% \sum \sigma_i \res^{-1}(\ket{U(i)}) \otimes \res^{-1}(\ket{V(i)})^{\star}
%\end{equation}

%%%%%%%%%%%%%%%%%%%%%%%%%%%%%%%%%%%%%%%%%%%%%%%%%%%%%%%%%%%%%%%%%%%%%%%%%%%%%%%%
\subsubsection{Partial trace and transposition (PG)}
%%%%%%%%%%%%%%%%%%%%%%%%%%%%%%%%%%%%%%%%%%%%%%%%%%%%%%%%%%%%%%%%%%%%%%%%%%%%%%%%
\paragraph{Partial trace}
Let $A\in \Lin(\Hilb{m_1}\otimes \Hilb{m_2}\otimes \ldots \otimes \Hilb{m_n})$,
the \emph{partial trace} of $A$ over subsystems indexed with indices 
$I=\{i_1, i_2, \ldots, i_k\}\subset \{1,2,\ldots, n\}$, where $k\leq n$, is 
operator $B\in \Lin(\bigotimes_{i\notin I} \Hilb{m_i})$, such that
$B=(\bigotimes_{i=1}^{n} \mathrm{op}(i))(A)$, where $\mathrm{op}(i)=\tr$ if 
$i\in I$ and $\mathrm{op}(i)=\1_{m_i}$ if $i\notin I$.

To perform the partial trace of an operator $A$, defined as above, in the 
package \qi{} one can call function
\begin{equation}
B=\fname{PartialTrace}[A, \{m_1, m_2, \ldots, m_n\}, \{i_1, i_2, 
\ldots, i_k\}].
\end{equation} 

\paragraph{Partial transpose}
Let $A\in \Lin(\Hilb{m_1}\otimes \Hilb{m_2}\otimes \ldots \otimes \Hilb{m_n})$,
the \emph{partial transpose} of $A$ with respect to subsystems indexed with 
indices $I=\{i_1, i_2, \ldots, i_k\}\subset \{1,2,\ldots, n\}$, where $k\leq 
n$, is the operator $B\in \Lin(\Hilb{m_1}\otimes \Hilb{m_2}\otimes \ldots 
\otimes \Hilb{m_n})$, such that $B=(\bigotimes_{i=1}^{n} \mathrm{op}(i))(A)$, 
where $\mathrm{op}(i)=\cdot^{\mathrm T}$ if $i\in I$ and 
$\mathrm{op}(i)=\1_{m_i}$ if $i\notin I$.

To perform the partial transpose of an operator $A$, defined as above, in the 
package \qi{} one can call function
\begin{equation}
B=\fname{PartialTranspose}[A, \{m_1, m_2, \ldots, m_n\}, \{i_1, i_2, 
\ldots, i_k\}].
\end{equation} 

%%%%%%%%%%%%%%%%%%%%%%%%%%%%%%%%%%%%%%%%%%%%%%%%%%%%%%%%%%%%%%%%%%%%%%%%%%%%%%%%
\subsection{Geometric tools}
%%%%%%%%%%%%%%%%%%%%%%%%%%%%%%%%%%%%%%%%%%%%%%%%%%%%%%%%%%%%%%%%%%%%%%%%%%%%%%%%

%%%%%%%%%%%%%%%%%%%%%%%%%%%%%%%%%%%%%%%%%%%%%%%%%%%%%%%%%%%%%%%%%%%%%%%%%%%%%%%%
\subsubsection{Distance measures}
%%%%%%%%%%%%%%%%%%%%%%%%%%%%%%%%%%%%%%%%%%%%%%%%%%%%%%%%%%%%%%%%%%%%%%%%%%%%%%%%
In \qi{} following distance measures between density matrices are implemented.
\paragraph{Trace distance} \ldots 

\begin{equation}
D_{\tr}(\rho_1,\rho_2)=\frac12\tr|\rho_1-\rho_2|
\end{equation}
In \qi{} the function $\fname{TraceDistance}[\rho_1,\rho_2]$ calculates the 
value of $D_{\tr}(\rho_1,\rho_2)$.

\paragraph{Fidelity}  \ldots
\begin{equation}
F(\rho_1,\rho_2)=
\left(\tr\sqrt{\sqrt{\rho_1}\rho_2\sqrt{\rho_1}}\right)^2
\end{equation}
In \qi{} the function $\fname{Fidelity}[\rho_1,\rho_2]$ calculates the 
value of $F(\rho_1,\rho_2)$.

\paragraph{Subfidelity}
Superfidelity is an lower bound on fidelity 
\cite{miszczak2008sup}
and is given by the following equation
\begin{equation}
E(\rho_1,\rho_2)=
\tr\rho_1\rho_2+\sqrt{2[(\tr\rho_1\rho_2)^2-\tr\rho_1\rho_2\rho_1\rho_2)]}
\end{equation}
In \qi{} the function $\fname{Subfidelity}[\rho_1,\rho_2]$ calculates the 
value of $E(\rho_1,\rho_2)$.

\paragraph{Superfidelity} Superfidelity is an upper bound on fidelity,
it is given by the following equation
\begin{equation}
G(\rho_1,\rho_2)=\tr\rho_1\rho_2+\sqrt{1-\tr\rho_1^2}\sqrt{1-\tr\rho_2^2}
\end{equation}
In \qi{} the function $\fname{Superfidelity}[\rho_1,\rho_2]$ calculates the 
value of $G(\rho_1,\rho_2)$.

%%%%%%%%%%%%%%%%%%%%%%%%%%%%%%%%%%%%%%%%%%%%%%%%%%%%%%%%%%%%%%%%%%%%%%%%%%%%%%%%
\subsubsection{Bloch Representation}
%%%%%%%%%%%%%%%%%%%%%%%%%%%%%%%%%%%%%%%%%%%%%%%%%%%%%%%%%%%%%%%%%%%%%%%%%%%%%%%%

%%%%%%%%%%%%%%%%%%%%%%%%%%%%%%%%%%%%%%%%%%%%%%%%%%%%%%%%%%%%%%%%%%%%%%%%%%%%%%%%
\subsection{Quantum states}
%%%%%%%%%%%%%%%%%%%%%%%%%%%%%%%%%%%%%%%%%%%%%%%%%%%%%%%%%%%%%%%%%%%%%%%%%%%%%%%%

%%%%%%%%%%%%%%%%%%%%%%%%%%%%%%%%%%%%%%%%%%%%%%%%%%%%%%%%%%%%%%%%%%%%%%%%%%%%%%%%
\subsubsection{Parametrizations}
%%%%%%%%%%%%%%%%%%%%%%%%%%%%%%%%%%%%%%%%%%%%%%%%%%%%%%%%%%%%%%%%%%%%%%%%%%%%%%%%
\qi{} package provides several parametrizations of quantum states. 

\paragraph{Pure states}
Any pure $N$-dimensional quantum state vector can be parametrized by set of 
$2N-2$ parameters \cite{vedra98entanglement}: $N-1$ parameters $\{\phi_1, 
\ldots, 
\phi_{N-1}\}$ parametrize probability vector according to the following equation
\begin{equation}
p_i=\sin^2\phi_{i-1} \prod\limits_{j=i}^{N-1}\cos^2\phi_j\text{\ \ with 
}\phi_0=\pi/2,
\end{equation}
and further $N-1$ parameters $\{\theta_1, \ldots
\theta_{N-1}\}$ parametrize relative phases. The resulting pure state 
is of the form
\begin{equation}
\ket{\psi}=
\left(
\begin{array}{c}
\sqrt{p_0}\\
\sqrt{p_1}e^{\ii \theta_1}\\
\vdots\\
\sqrt{p_{N-1}}e^{\ii \theta_{N-1}}
\end{array}
\right).
\end{equation}
This parametrization is implemented in \qi{} in function
$\fname{StateVector}[l]$ where $l=\{\phi_1, \ldots, \phi_{N-1}, \theta_1, \ldots
\theta_{N-1}\}$.

\paragraph{Special unitary matrices}
$d$-dimensional special unitary matrix can be parametrized using 
$d-1$-parameters. In \qi{} function 
$\fname{SpecialUnitary}[d,\{l_1,\ldots,l_{d-1}\}]$ 
implements Euler angles parametrization introduced in 
\cite{zyczkowski94random}. Parameters $l_i$ are 
truncated so that they belong to the interval $l_i\in [0,1]$.
%%%%%%%%%%%%%%%%%%%%%%%%%%%%%%%%%%%%%%%%%%%%%%%%%%%%%%%%%%%%%%%%%%%%%%%%%%%%%%%%
\subsubsection{Random states}
%%%%%%%%%%%%%%%%%%%%%%%%%%%%%%%%%%%%%%%%%%%%%%%%%%%%%%%%%%%%%%%%%%%%%%%%%%%%%%%%

%%%%%%%%%%%%%%%%%%%%%%%%%%%%%%%%%%%%%%%%%%%%%%%%%%%%%%%%%%%%%%%%%%%%%%%%%%%%%%%%
\subsection{Quantum channels}
%%%%%%%%%%%%%%%%%%%%%%%%%%%%%%%%%%%%%%%%%%%%%%%%%%%%%%%%%%%%%%%%%%%%%%%%%%%%%%%%

%%%%%%%%%%%%%%%%%%%%%%%%%%%%%%%%%%%%%%%%%%%%%%%%%%%%%%%%%%%%%%%%%%%%%%%%%%%%%%%%
\section{Examples}
%%%%%%%%%%%%%%%%%%%%%%%%%%%%%%%%%%%%%%%%%%%%%%%%%%%%%%%%%%%%%%%%%%%%%%%%%%%%%%%%

%%%%%%%%%%%%%%%%%%%%%%%%%%%%%%%%%%%%%%%%%%%%%%%%%%%%%%%%%%%%%%%%%%%%%%%%%%%%%%%%
\section{Final remarks}
%%%%%%%%%%%%%%%%%%%%%%%%%%%%%%%%%%%%%%%%%%%%%%%%%%%%%%%%%%%%%%%%%%%%%%%%%%%%%%%%

%%%%%%%%%%%%%%%%%%%%%%%%%%%%%%%%%%%%%%%%%%%%%%%%%%%%%%%%%%%%%%%%%%%%%%%%%%%%%%%%
\section{Acknowledgements}
%%%%%%%%%%%%%%%%%%%%%%%%%%%%%%%%%%%%%%%%%%%%%%%%%%%%%%%%%%%%%%%%%%%%%%%%%%%%%%%%
This work was supported by the Polish National Science Centre under the research
project \dots. Authors would like to thank K.~\.Zyczkowski for interesting
discussions.

\bibliographystyle{plain}
\bibliography{qi}

\end{document}
