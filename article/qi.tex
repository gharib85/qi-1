% v. 0.2 - 11/08/2009 - Jarek - extended introductions and bibliography
% v. 0.21 - 12/08/2009 - Jarek - preliminary sections
% v. 0.22 - 13/08/2009 - Jarek - added in intro and sec. 2
% v. 0.3 - 17/11/2009 - Jarek - added an appendix with a list of functions
% v. 0.31 - 18/11/2009 - Jarek - sections on states and operations
% v. 0.32 - 19/11/2009 - Jarek - improved abstact, applied CPC template 
% v. 0.33 - 24/09/2010 - Jarek - update basic info, sync with 0.3.18 version 
\documentclass{elsart}

\usepackage{hyperref}

%% This list environment is used for the references in the
%% Program Summary
%%
\newcounter{bla}
\newenvironment{refnummer}{%
\list{[\arabic{bla}]}%
{\usecounter{bla}%
 \setlength{\itemindent}{0pt}%
 \setlength{\topsep}{0pt}%
 \setlength{\itemsep}{0pt}%
 \setlength{\labelsep}{2pt}%
 \setlength{\listparindent}{0pt}%
 \settowidth{\labelwidth}{[9]}%
 \setlength{\leftmargin}{\labelwidth}%
 \addtolength{\leftmargin}{\labelsep}%
 \setlength{\rightmargin}{0pt}}}
 {\endlist}

%%%%%%%%%%%%%%%%%%%%%%%%%%%%%%%%%%%%%%%%%%%%%%%%%%%%%%%%%%%%%%%%%%%%%%%%%%%%%%%%
\usepackage{amsmath,amssymb}
\usepackage{dsfont}
\newcommand{\ket}[1]{\ensuremath{|#1\rangle}}
\newcommand{\bra}[1]{\ensuremath{\langle#1|}}
\newcommand{\Mathematica}{\emph{Mathematica}}
\newcommand{\wek}{\mathbf{vec}}
\newcommand{\res}{\mathbf{res}}
\newcommand{\1}{{\rm 1\hspace{-0.9mm}l}}
\newcommand{\Id}{\1}
\newcommand{\SWAP}{\ensuremath{\mathrm{SWAP}}}
\newcommand{\tr}{\mathrm{tr}}
\newcommand{\M}{\ensuremath{\mathbb{M}}}
\newcommand{\qi}{QI}
\newcommand{\code}[1]{\texttt{\small #1}}
\newcommand{\HS}[1]{\ensuremath{\mathcal{#1}}} % Hilbert space
\newcommand{\Cplx}{\ensuremath{\mathbb{C}}}
\newcommand{\eg}{\emph{eg.}}
\newcommand{\ie}{\emph{ie.}}
%\newcommand{\etal}{\emph{et al.}}
\newcommand{\Prob}[1]{\ensuremath{\mathrm{P}(#1)}}
\newcommand{\Observ}[1]{\ensuremath{#1}}
\newcommand{\Spectrum}[1]{\ensuremath{\sigma(#1)}}
\newcommand{\Spec}[1]{\Spectrum{#1}}
\newcommand{\ketbra}[2]{\ensuremath{\ket{#1}\bra{#2}}}
\newcommand{\proj}[1]{\ensuremath{\ketbra{#1}{#1}}}
\newcommand{\Proj}[1]{\proj{#1}}
\newcommand{\iner}[2]{\braket{#1}{#2}}
\newcommand{\Iner}[2]{\iner{#1}{#2}}
\newcommand{\braket}[2]{\ensuremath{\langle#1|#2\rangle}}
\newcommand{\Tr}[2][]{\ensuremath{\tr_{#1}{#2}}}
\newcommand{\id}{\mathds{1}}
\newcommand{\Space}[1]{\mathcal{#1}}
\newcommand{\SetOfStates}[1]{\ensuremath{\mathcal{S}(#1)}}
\newcommand{\States}[1]{\SetOfStates{#1}}
\newcommand{\Real}{\ensuremath{\mathds{R}}}
\newcommand{\N}{\ensuremath{\mathds{N}}}
\newcommand{\Z}{\ensuremath{\mathds{Z}}}
\newcommand{\set}[2]{\ensuremath{\left\{#1|#2\right\}}}
\newcommand{\re}[1]{\ensuremath{\mathrm{Re}\left(#1\right)}}
\newcommand{\im}[1]{\ensuremath{\mathrm{Im}\left(#1\right)}}
\newcommand{\Group}[2]{\ensuremath{(#1,#2)}}
\newcommand{\ord}[2]{\mathrm{ord}_#2(#1)}
\newcommand{\Var}[1]{\ensuremath{\mathrm{Var}(#1)}}
\newcommand{\halmos}{\newline\vspace{3mm}\hfill $\Box$}
\newcommand{\proof}{\noindent {\it Proof.\ }}
\newtheorem{theorem}{Theorem}

\usepackage{xcolor}
\newcommand{\todo}[1]{\textcolor{red}{\bf TODO: #1}}
%%%%%%%%%%%%%%%%%%%%%%%%%%%%%%%%%%%%%%%%%%%%%%%%%%%%%%%%%%%%%%%%%%%%%%%%%%%%%%%%

\journal{Computer Physics Communications}

\begin{document}

\begin{frontmatter}

%% Title, authors and addresses

%% use the tnoteref command within \title for footnotes;
%% use the tnotetext command for theassociated footnote;
%% use the fnref command within \author or \address for footnotes;
%% use the fntext command for theassociated footnote;
%% use the corref command within \author for corresponding author footnotes;
%% use the cortext command for theassociated footnote;
%% use the ead command for the email address,
%% and the form \ead[url] for the home page:
%% \title{Title\tnoteref{label1}}
%% \tnotetext[label1]{}
%% \author{Name\corref{cor1}\fnref{label2}}
%% \ead{email address}
%% \ead[url]{home page}
%% \fntext[label2]{}
%% \cortext[cor1]{}
%% \address{Address\fnref{label3}}
%% \fntext[label3]{}

\title{\qi: A package for the analysis of quantum states and operations in
\Mathematica}

\date{24/09/2009 (v. 0.33)}

\author{J.~A.~Miszczak\thanksref{author}},
\ead{miszczak@iitis.pl}
\author{P.~Gawron},
\author{Z.~Pucha{\l}a}

\thanks[author]{Corresponding author}

\address{Institute of Theoretical and Applied Informatics, Polish Academy of 
Sciences, Ba{\l}tycka 5, 44-100 Gliwice, Poland}

\begin{abstract}
QI is a package of functions for the \Mathematica\ computer algebra system,
which provides a framework for the analysis of quantum states and quantum
operations. In contrast to many available packages for symbolic and numerical
simulation of quantum computation presented package is focused on geometrical
aspects of quantum information theory. In particular \qi provides
parametrization of quantum states and selected families of quantum operation,
function for constructing composite quantum operations and methods for the
generation and analysis of random quantum operations. Also basic structures are
provided including the construction of ket vectors, basic unitary gates, random
states and unitaries, distance measures between quantum states and the selected
methods for the analysis of separability in quantum systems.

\begin{flushleft}
  %Insert your suggested PACS number here
%quantum information, symbolic computation (computer algebra) 
PACS: 03.67.-a; 02.70.Wz.
\end{flushleft}

\begin{keyword}
  % Please give some freely chosen keywords that we can use in a
  % cumulative keyword index.
Quantum states; Quantum operations; Partial operations.
\end{keyword}
\end{abstract}

\end{frontmatter}
%%%%%%%%%%%%%%%%%%%%%%%%%%%%%%%%%%%%%%%%%%%%%%%%%%%%%%%%%%%%%%%%%%%%%%%%%%%%%%%%
{\bf PROGRAM SUMMARY}
  %Delete as appropriate.

\begin{small}
\noindent
{\em Manuscript Title:} \qi: A package for the analysis of quantum states and operations in
\Mathematica\\
{\em Authors:} J.A.~Miszczak, P.~Gawron, Z.~Pucha{\l}a \\
{\em Program Title:} QI \\
{\em Journal Reference:}                                      \\
  %Leave blank, supplied by Elsevier.
{\em Catalogue identifier:}                                   \\
  %Leave blank, supplied by Elsevier.
{\em Licensing provisions:} GPLv3 \\
{\em Programming language:} Mathematica 7\\
{\em Computer:} Any computer supporting Mathematica 7\\
  %Computer(s) for which program has been designed.
{\em Operating system:} Any operating system capable of running Mathematica 7 or higher, \eg\ GNU/Linux, MacOS X, FreeBSD, Microsoft Windows XP\\
  %Operating system(s) for which program has been designed.
{\em RAM:} bytes                                              \\
  %RAM in bytes required to execute program with typical data.
{\em Number of processors used:}                              \\
  %If more than one processor.
{\em Supplementary material:}                                 \\
  % Fill in if necessary, otherwise leave out.
{\em Keywords:} quantum states, quantum operations, partial operations  \\
  % Please give some freely chosen keywords that we can use in a
  % cumulative keyword index.
{\em PACS:} 03.67.-a, 02.70.Wz.\\
  % see http://www.aip.org/pacs/pacs.html
{\em Classification:} 4.15 \\
  %Classify using CPC Program Library Subject Index, see (
  % http://cpc.cs.qub.ac.uk/subjectIndex/SUBJECT_index.html)
  %e.g. 4.4 Feynman diagrams, 5 Computer Algebra.
{\em External routines/libraries:}                                      \\
  % Fill in if necessary, otherwise leave out.
{\em Subprograms used:}                                       \\
  %Fill in if necessary, otherwise leave out.

{\em Nature of problem:}\\
  %Describe the nature of the problem here.
  Construction of composed quantum operations, analysis of quantum states and
  operations.
   \\
{\em Solution method:}\\
  %Describe the method solution here.
  A package of functions for \Mathematica\ computer algebra system.
   \\
{\em Restrictions:}\\
  %Describe any restrictions on the complexity of the problem here.
  Running time of the presented procedures grows rapidliy with the dimensionalty
  of the problem.
   \\
%{\em Unusual features:}\\
%  %Describe any unusual features of the program/problem here.
%   \\
%{\em Additional comments:}\\
%  %Provide any additional comments here.
%   \\
{\em Running time:}\\
  %Give an indication of the typical running time here.
   \\

\end{small}

\newpage

% In program descriptions the main text of the paper is listed under
% the heading LONG WRITE-UP.

\hspace{1pc}
{\bf LONG WRITE-UP}
%%%%%%%%%%%%%%%%%%%%%%%%%%%%%%%%%%%%%%%%%%%%%%%%%%%%%%%%%%%%%%%%%%%%%%%%%%%%%%%%


\tableofcontents

%% \linenumbers

%% main text
%%%%%%%%%%%%%%%%%%%%%%%%%%%%%%%%%%%%%%%%%%%%%%%%%%%%%%%%%%%%%%%%%%%%%%%%%%%%%%%%
\section{Introduction}\label{sec:intro}
%%%%%%%%%%%%%%%%%%%%%%%%%%%%%%%%%%%%%%%%%%%%%%%%%%%%%%%%%%%%%%%%%%%%%%%%%%%%%%%%
Quantum information theory aims to harness the behavior of quantum mechanical
objects to store, transfer and process information~\cite{hayashi}. This behavior
is in many cases very different from the one we observe in classical world.
Quantum algorithms and protocols take advantage of superposition of states and
require the presence of entangled states. Both phenomena arise from the rich
structure of the space of quantum states. Hence, to explore the capabilities of
quantum information processing, one needs to fully understand this
space~\cite{BZ06}. 

Quantum mechanics provides us also with much larger than in classical case space
of allowed operations which can be used to manipulate quantum
states~\cite{hayashi,BZ06}. Recent results concerning additivity
problems~\cite{hastings09superadditivity} show that we are far from full
understanding the nature of quantum channels. Exploring the space of quantum
operations is fascinating, but cumbersome task.

We present a~package of functions developed for \Mathematica\ computing system
which aims to simplify the analysis of quantum states and quantum operations.
The package was developed in simplicity in mind and thus it uses only basic data
structures available in \Mathematica. This allows to relatively easily port
implemented functions to other scientific software systems. Also, in contrast to
most quantum computing packages
available~\cite{qdensity,qucalc,quantum2,qcwave}, \qi\ is not aimed to provide
tool for simulating quantum algorithms and protocols. We rather focus on the
analysis of quantum states used in those protocols and quantum channels, which
are used to describe allowed physical operations. The main goal of presented
package is to provide basic mathematical tools useful for studding geometrical
properties of quantum stats and quantum channels.

This paper is organized as follows. In Section~\ref{sec:qi-intro} we review
basic notion of quantum density matrices and quantum channels representing
allowed physical transformations of density matrices. Section~\ref{sec:over}
presents an overview of functionality provided by \qi\ package and describes
some operations used as building blocks in specialized functions.
Sections~\ref{sec:states} and \ref{sec:channels} provide detailed description of
functions implemented in~\qi. Finally Section~\ref{sec:comclude} provides some
concluding remarks.

%%%%%%%%%%%%%%%%%%%%%%%%%%%%%%%%%%%%%%%%%%%%%%%%%%%%%%%%%%%%%%%%%%%%%%%%%%%%%%%%
\section{Quantum states and operations}\label{sec:qi-intro}
%%%%%%%%%%%%%%%%%%%%%%%%%%%%%%%%%%%%%%%%%%%%%%%%%%%%%%%%%%%%%%%%%%%%%%%%%%%%%%%%
Let $\HS{H}$ be a~separable, complex Hilbert space used to described the system
in question. In quantum information theory we deal mainly with
finite-dimensional Hilbert spaces, so usually we have to are in situation where
$\HS{H}=\Cplx^n$. State of the system is described by the density matrix, \ie\
operator $\rho:\HS{H}\rightarrow\HS{H}$, which is positive ($\rho\geq0$) and
normalized ($\tr{\rho}=1$). 

%%%%%%%%%%%%%%%%%%%%%%%%%%%%%%%%%%%%%%%%%%%%%%%%%%%%%%%%%%%%%%%%%%%%%%%%%%%%%%%%
\subsection{Pure and mixed states}
%%%%%%%%%%%%%%%%%%%%%%%%%%%%%%%%%%%%%%%%%%%%%%%%%%%%%%%%%%%%%%%%%%%%%%%%%%%%%%%%
We use Dirac notation \cite{dirac58principles} for inner product in $\HS{H}$
\begin{equation}
  \Iner{\psi}{\phi},
\label{eqn:dirac-iner}
\end{equation}
where $\ket{\psi},\ket{\phi}\in\HS{H}$ and $\bra{\phi} = \ket{\phi}^\star$ 
is a~complex conjugate of $\ket{\phi}$. Symbol $\ket{\psi}\bra{\phi}$ denotes
the operator of rank one (\ie\ a projection operator) which acts on a vector
$\ket{\alpha}\in\HS{H}$ as
\begin{equation}
\left(\ket{\psi}\bra{\phi}\right)\ket{\alpha} = \Iner{\phi}{\alpha}\ket{\psi}.
\label{eqn:dirac-proj}
\end{equation}

In quantum information theory we deal mainly with finite-dimensional Hilbert
spaces.

%%%%%%%%%%%%%%%%%%%%%%%%%%%%%%%%%%%%%%%%
\subsection{Observables}
%%%%%%%%%%%%%%%%%%%%%%%%%%%%%%%%%%%%%%%%
The term \emph{observable}\index{observable} is used to describe a physical
quantity of a system which can be observed and measured. Observables are
quantum mechanical analogous of the \emph{random variables}\index{random
variable} from classical mechanics.

In quantum mechanics observables are described by self-adjoint operators,
\ie\ linear functions $X:\HS{H}\mapsto\HS{H}$ such that
\begin{equation}
\Iner{X\psi}{\phi}=\Iner{\psi}{X^\star\phi},
\label{eqn:selfadj}
\end{equation}
for any $\ket{\phi},\ket{\psi}\in\HS{H}$. An important property of self-adjoint
operators is expressed by the following theorem.

\begin{theorem}[Spectral decomposition]
Every self-adjoint operator $\Observ{A}$ can be decomposed according to the
formula
\begin{equation}
\Observ{A} = \sum_{\lambda_i\in\Spec{\Observ{A}}} \lambda_i\proj{x_i}
\end{equation}
with $\sum_i\Proj{x_i}=\Id$, where $\ket{x_i}$ are the eigenvectors of
$\Observ{A}$ with corresponding eigenvalues $\lambda_i$.
\end{theorem}
Here $\sigma(A)$ denotes the spectrum of the operator $A$. The mapping
$\mu:\lambda_i\mapsto \Proj{x_i}$ is called \emph{spectral
measure} \cite{mlak}.\index{spectral measure}

Spectral decomposition can be used to define functions on the space of
self-adjoint operators. If $f:\Real\mapsto\Cplx$ is any function, one can define
$f(\Observ{A})$ as
\begin{equation}
 f(A) = \sum_{\lambda_i\in\Spec{\Observ{A}}} f(\lambda_i)\proj{x_i}.
\end{equation}
For example, if we take $A=NOT$, we have
\begin{equation}
 NOT =  +1\left(\begin{array}{cc}
 		\frac{1}{2} & \frac{1}{2}\\
 		\frac{1}{2} & \frac{1}{2}
        \end{array}\right)
		-1 \left(\begin{array}{cc}
                          \frac{1}{2} & -\frac{1}{2}\\
                          -\frac{1}{2} & \frac{1}{2}
                         \end{array}\right),
\end{equation}
and square root of $NOT$ can be easily calculated as
\begin{equation}
 \sqrt{NOT} = \sqrt{+1}\left(\begin{array}{cc}
 		\frac{1}{2} & \frac{1}{2}\\
 		\frac{1}{2} & \frac{1}{2}
        \end{array}\right)
		+\sqrt{-1} \left(\begin{array}{cc}
                          \frac{1}{2} & -\frac{1}{2}\\
                          -\frac{1}{2} & \frac{1}{2}
                         \end{array}\right).
\end{equation}

%%%%%%%%%%%%%%%%%%%%%%%%%%%%%%%%%%%%%%%%
\subsection{States}
%%%%%%%%%%%%%%%%%%%%%%%%%%%%%%%%%%%%%%%%
In quantum mechanics the state of the system is described by the density matrix,
\ie\ Hermitian operator $\rho:\HS{H}\rightarrow\HS{H}$, which is
\begin{equation}
 \rho\geq0\ \ \mathrm{(positive)}
\end{equation}
and
\begin{equation}
 \tr{\rho}=1\ \ \mathrm{(normalized)}.
\end{equation}
Density matrix in the analogue of the classical \emph{probability
distribution}. \index{probability distribution}

The set of states $\SetOfStates{\mathcal{H}}$ is a~convex set and thus every
$\rho\in\SetOfStates{\mathcal{H}}$ can be represented as a~convex combination
\begin{equation}
\rho=\sum_{i}p_i\sigma_i,
\end{equation}
with $\sigma_i\in\SetOfStates{\mathcal{H}},\ \sum_{i}p_i=1$. 

If the operator $\rho$ additionally fulfils the condition
\begin{equation}
 \rho^2 = \rho
\end{equation}
\ie\ $\rho$ is a~projection operator, the state described by $\rho$ is
said to be \emph{pure}.\index{pure state} In this case density operator posesses
only one eigenvector $\ket{\alpha}$, which can be used to describe the state.
Therefore, in the case of pure states, we can write
\begin{equation}
  \rho = \proj{\alpha}.
\end{equation}

In particular, if $\HS{H}=\Cplx^2$ we say that $\ket{\psi}\in\HS{H}$ describes
the state of a qubit. In this case
\begin{equation}
 \ket{\psi} = \alpha \ket{0} + \beta \ket{1},
\end{equation}
where $\alpha^2+\beta^2=1$ and $\ket{0}, \ket{1}\in \Cplx^2$ form a base
$\Cplx^2$. 

%%%%%%%%%%%%%%%%%%%%%%%%%%%%%%%%%%%%%%%%
\subsection{Unitary evolution}\label{sec:math-unit-evol}
%%%%%%%%%%%%%%%%%%%%%%%%%%%%%%%%%%%%%%%%
Let us assume that the described system is isolated during the time of
evolution. If initially the system is in a pure state $\ket{\psi}$, then the
evolution of the system is given by some unitary operator $U$. The action of the
operator $U$ on the initial state is given by a formula
\begin{equation}
 \ket{\psi} \mapsto U \ket{\psi}.
\end{equation}
If the initial state of the system is mixed and given by the density matrix
$\rho$, then the final state is given by
\begin{equation}\label{eqn:evol-mixed}
 \rho \mapsto U\rho U^\dagger.
\end{equation}

However, in many situations it is impossible to avoid the interaction of the
systems with an environment.


%%%%%%%%%%%%%%%%%%%%%%%%%%%%%%%%%%%%%%%%%%%%%%%%%%%%%%%%%%%%%%%%%%%%%%%%%%%%%%%%
\subsection{Quantum processes}
%%%%%%%%%%%%%%%%%%%%%%%%%%%%%%%%%%%%%%%%%%%%%%%%%%%%%%%%%%%%%%%%%%%%%%%%%%%%%%%%
The most general form of the evolution of a quantum system is given in terms of
quantum channels. In this paper we consider quantum channels which are 
Completely Positive Trace Preserving (CP-TP) maps. 

In order an map $\Phi$ to be a CP-TP map it has to fulfill set of the following
conditions:
\begin{enumerate}
\item It has to preserve trace, positivity and hermiticity, \ie\
  \begin{equation}
	  \Tr{\Phi(\rho)}=1, \Phi(\rho)\geq0\ \mathrm{and\ } \Phi(\rho)=\Phi(\rho)^\dagger.
  \end{equation}
\item It has to be linear
  \begin{equation}
	  \Phi\left(\sum_i p_i\rho_i \right)=\sum_i p_i \Phi\left(\rho_i \right).
  \end{equation}
\item Finally it has to be \emph{completely positive}, \ie\ for $\rho^{(n)} \geq
  0$ we require that
  \begin{eqnarray}
	   (\Phi\otimes\id_n) \rho^{(n)} \geq 0,\ n\in N,
  \end{eqnarray}
  where $\rho^{(n)}$ is an element of an appropriate space of 
  states.
\end{enumerate}
These conditions are required for $\Phi$ to preserve the set of quantum states.

In the most general case quantum evolution is described by a superoperator
$\Phi$, acting on $\mathcal{M}_{N}$, which can be expressed in Kraus form
\cite{BZ06,hayashi}
\begin{equation}
\Phi(\rho)=\sum_k E_k^{\phantom{\dagger}} \rho E_k^\dagger,
\end{equation}
where $\sum_k E_k^\dagger E_k^{\phantom{\dagger}}=\id$.

Alternatively quantum operations an be represented by a \emph{superoperator
matrix} $M_\Phi$. The superoperator matrix is a~representation of linear
operator in the canonical basis. The following formula allows to transform set
of Kraus operators $\{E_k\}$ into superoperator matrix $M_\Phi$~\cite[Ch.
10]{BZ06}
\begin{equation} \label{r:Mphi}
	M_\Phi=\sum_{k=1}^{N^2} E_k \otimes E^\ast_k, 
\end{equation}
where $N = \dim(E_k)$ and `$\ast$' denotes element-wise complex conjugation.

The dynamical matrix for the operations $\Phi$ is defined as $D_\Phi=M_\Phi^R$,
where `${}^R$' denotes a \emph{reshuffling} operation~\cite{BZ06}. The dynamical matrix
for the trace preserving operation acting on $N$-dimensional system is
an~$N^2\times N^2$ positive defined matrix with trace~$N$. We can introduce
natural correspondence between such matrices and density matrices on $N^2$ by
normalizing $D_\Phi$. Such a correspondence is known as \emph{Jamio{\l}kowski
isomorphism}~\cite{jamiolkowski72linear,zyczkowski04duality}.

Let $\Phi$ be a~completely positive trace preserving map acting on density
matrices. We define Jamio\l{}kowski matrix of $\Phi$ as
\begin{equation} \label{r:jam}
 \rho_\Phi = \frac{1}{N}D_\Phi. 
\end{equation}

Jamio\l{}kowski matrix has the same mathematical properties as a quantum state
\ie{} it is a semi-definite positive matrix with trace equal to one. It is
sometimes referred to as \emph{Jamio\l{}kowski state matrix}.



%%%%%%%%%%%%%%%%%%%%%%%%%%%%%%%%%%%%%%%%%%%%%%%%%%%%%%%%%%%%%%%%%%%%%%%%%%%%%%%%
\section{Overview of \qi}\label{sec:over}
%%%%%%%%%%%%%%%%%%%%%%%%%%%%%%%%%%%%%%%%%%%%%%%%%%%%%%%%%%%%%%%%%%%%%%%%%%%%%%%%
\todo{troche o zalozeniach projektowych czyli dlaczego \qi\ jest inne niz
pozostale pakiety}

%%%%%%%%%%%%%%%%%%%%%%%%%%%%%%%%%%%%%%%%%%%%%%%%%%%%%%%%%%%%%%%%%%%%%%%%%%%%%%%%
\subsection{Design principles}
%%%%%%%%%%%%%%%%%%%%%%%%%%%%%%%%%%%%%%%%%%%%%%%%%%%%%%%%%%%%%%%%%%%%%%%%%%%%%%%%

\newcounter{principle}
\begin{list}{\textbf{P\arabic{principle}}}{\usecounter{principle}}
\item Functions are as simple as possible -- functions have minimum reasonable
number of arguments and implement small pieces of functionality. For example in
order to get partial trace on few subsystems user needs to perform permutation
and then apply partial trace 
\item User knows what she is doing -- functions implemented in \qi\ do not
validate the input and user is expected to provide reasonable data.
\item Only basic \Mathematica\ structures are used -- functions operator on
plain \Mathematica\ lists and return lists.
\item Some data are used more often then other -- \qi\ predefines some commonly
used matrices, for example matrices for \SWAP\ operation and Pauli matrices for
small dimensions.
\end{list}

%%%%%%%%%%%%%%%%%%%%%%%%%%%%%%%%%%%%%%%%%%%%%%%%%%%%%%%%%%%%%%%%%%%%%%%%%%%%%%%%
\subsection{Basic algebraic operations}
%%%%%%%%%%%%%%%%%%%%%%%%%%%%%%%%%%%%%%%%%%%%%%%%%%%%%%%%%%%%%%%%%%%%%%%%%%%%%%%%

%%%%%%%%%%%%%%%%%%%%%%%%%%%%%%%%%%%%%%%%%%%%%%%%%%%%%%%%%%%%%%%%%%%%%%%%%%%%%%%%
\subsubsection{Kronecker product}
%%%%%%%%%%%%%%%%%%%%%%%%%%%%%%%%%%%%%%%%%%%%%%%%%%%%%%%%%%%%%%%%%%%%%%%%%%%%%%%%
Version 6 of \Mathematica\ introduced \code{KroneckerProduct} function.

%%%%%%%%%%%%%%%%%%%%%%%%%%%%%%%%%%%%%%%%%%%%%%%%%%%%%%%%%%%%%%%%%%%%%%%%%%%%%%%%
\subsection{Comonly used matrices}
%%%%%%%%%%%%%%%%%%%%%%%%%%%%%%%%%%%%%%%%%%%%%%%%%%%%%%%%%%%%%%%%%%%%%%%%%%%%%%%%
\qi\ predefines set of commonly used matrices. Among avaible matrices one can
find
\begin{itemize}
\item Pauli matrices
\begin{equation}
\left(
\begin{smallmatrix}
 0 & 1 \\
 1 & 0
\end{smallmatrix}
\right),\left(
\begin{smallmatrix}
 0 & -i \\
 i & 0
\end{smallmatrix}
\right),\left(
\begin{smallmatrix}
 1 & 0 \\
 0 & -1
\end{smallmatrix}
\right).
\end{equation}

\item Gell-Mann matrices
\begin{equation}
\left(
\begin{smallmatrix}
 0 & 1 & 0 \\
 1 & 0 & 0 \\
 0 & 0 & 0
\end{smallmatrix}
\right),\left(
\begin{smallmatrix}
 0 & -i & 0 \\
 i & 0 & 0 \\
 0 & 0 & 0
\end{smallmatrix}
\right),
\left(
\begin{smallmatrix}
 1 & 0 & 0 \\
 0 & -1 & 0 \\
 0 & 0 & 0
\end{smallmatrix}
\right),
\left(
\begin{smallmatrix}
 0 & 0 & 1 \\
 0 & 0 & 0 \\
 1 & 0 & 0
\end{smallmatrix}
\right),
\left(
\begin{smallmatrix}
 0 & 0 & -i \\
 0 & 0 & 0 \\
 i & 0 & 0
\end{smallmatrix}
\right),
\left(
\begin{smallmatrix}
 0 & 0 & 0 \\
 0 & 0 & 1 \\
 0 & 1 & 0
\end{smallmatrix}
\right),
\left(
\begin{smallmatrix}
 0 & 0 & 0 \\
 0 & 0 & -i \\
 0 & i & 0
\end{smallmatrix}
\right),
\frac{1}{\sqrt{3}}\left(
\begin{smallmatrix}
 1 & 0 & 0 \\
 0 & 1 & 0 \\
 0 & 0 & -2
\end{smallmatrix}
\right).
\end{equation}
\item Hadamard matrix 
\begin{equation}
\left(
\begin{smallmatrix}
 \frac{1}{\sqrt{2}} & \frac{1}{\sqrt{2}} \\
 \frac{1}{\sqrt{2}} & -\frac{1}{\sqrt{2}}
\end{smallmatrix}
\right)
\end{equation}
\end{itemize}

%%%%%%%%%%%%%%%%%%%%%%%%%%%%%%%%%%%%%%%%%%%%%%%%%%%%%%%%%%%%%%%%%%%%%%%%%%%%%%%%
\section{Quantum states in \qi}\label{sec:states}
%%%%%%%%%%%%%%%%%%%%%%%%%%%%%%%%%%%%%%%%%%%%%%%%%%%%%%%%%%%%%%%%%%%%%%%%%%%%%%%%

%%%%%%%%%%%%%%%%%%%%%%%%%%%%%%%%%%%%%%%%%%%%%%%%%%%%%%%%%%%%%%%%%%%%%%%%%%%%%%%%
\subsection{Probability distributions}
%%%%%%%%%%%%%%%%%%%%%%%%%%%%%%%%%%%%%%%%%%%%%%%%%%%%%%%%%%%%%%%%%%%%%%%%%%%%%%%%
State in quantum mechanics is a generalization of classical probability
distribution.

%%%%%%%%%%%%%%%%%%%%%%%%%%%%%%%%%%%%%%%%%%%%%%%%%%%%%%%%%%%%%%%%%%%%%%%%%%%%%%%%
\subsection{One-qubit states}
%%%%%%%%%%%%%%%%%%%%%%%%%%%%%%%%%%%%%%%%%%%%%%%%%%%%%%%%%%%%%%%%%%%%%%%%%%%%%%%%

%%%%%%%%%%%%%%%%%%%%%%%%%%%%%%%%%%%%%%%%%%%%%%%%%%%%%%%%%%%%%%%%%%%%%%%%%%%%%%%%
\subsection{States in higher dimension}
%%%%%%%%%%%%%%%%%%%%%%%%%%%%%%%%%%%%%%%%%%%%%%%%%%%%%%%%%%%%%%%%%%%%%%%%%%%%%%%%
\qi\ provides function for cosntructing pure state in arbitrary dimension

Parametrization of pure states in higher dimensions is provided according to
\cite{tilma02generalized}.

%%%%%%%%%%%%%%%%%%%%%%%%%%%%%%%%%%%%%%%%%%%%%%%%%%%%%%%%%%%%%%%%%%%%%%%%%%%%%%%%
\subsection{Special states}
%%%%%%%%%%%%%%%%%%%%%%%%%%%%%%%%%%%%%%%%%%%%%%%%%%%%%%%%%%%%%%%%%%%%%%%%%%%%%%%%
\begin{itemize}
\item maximally entangled states
\item maximally mixed states
\item Werner states
\item Isotropic states
\end{itemize}

\todo{zapoznac sie ze stanami horodeckich z bound entanglement}
%%%%%%%%%%%%%%%%%%%%%%%%%%%%%%%%%%%%%%%%%%%%%%%%%%%%%%%%%%%%%%%%%%%%%%%%%%%%%%%%
\subsection{Distance measures}
%%%%%%%%%%%%%%%%%%%%%%%%%%%%%%%%%%%%%%%%%%%%%%%%%%%%%%%%%%%%%%%%%%%%%%%%%%%%%%%%


%%%%%%%%%%%%%%%%%%%%%%%%%%%%%%%%%%%%%%%%%%%%%%%%%%%%%%%%%%%%%%%%%%%%%%%%%%%%%%%%
\section{Quantum operations in \qi}\label{sec:channels}
%%%%%%%%%%%%%%%%%%%%%%%%%%%%%%%%%%%%%%%%%%%%%%%%%%%%%%%%%%%%%%%%%%%%%%%%%%%%%%%%

%%%%%%%%%%%%%%%%%%%%%%%%%%%%%%%%%%%%%%%%%%%%%%%%%%%%%%%%%%%%%%%%%%%%%%%%%%%%%%%%
\subsection{Representation of quantum operations}
%%%%%%%%%%%%%%%%%%%%%%%%%%%%%%%%%%%%%%%%%%%%%%%%%%%%%%%%%%%%%%%%%%%%%%%%%%%%%%%%

%%%%%%%%%%%%%%%%%%%%%%%%%%%%%%%%%%%%%%%%%%%%%%%%%%%%%%%%%%%%%%%%%%%%%%%%%%%%%%%%
\subsection{Random quantum operations}
%%%%%%%%%%%%%%%%%%%%%%%%%%%%%%%%%%%%%%%%%%%%%%%%%%%%%%%%%%%%%%%%%%%%%%%%%%%%%%%%
\cite{Bruzda2009320}

%%%%%%%%%%%%%%%%%%%%%%%%%%%%%%%%%%%%%%%%%%%%%%%%%%%%%%%%%%%%%%%%%%%%%%%%%%%%%%%%
\subsection{Partial operations}
%%%%%%%%%%%%%%%%%%%%%%%%%%%%%%%%%%%%%%%%%%%%%%%%%%%%%%%%%%%%%%%%%%%%%%%%%%%%%%%%
\qi\ implements partial operations using general method for constructing
channels acting on subsystems. This method is based on formula
\begin{equation}\label{eqn:def-tensor}
(\Phi\otimes\Id)(\rho) = 
\left(\res^{-1}\left(M_\Phi\res\left(\rho^R\right)\right)\right)^R
\end{equation}
where ${}^R$ denotes the reshuffling operation and $M_\Phi$ denotes the matrix
of the linear map $\Phi$
\begin{equation}
M_\Phi = \tr \epsilon_i \Phi(\epsilon_j),
\end{equation}
where $\{\epsilon_i\}_i=1,\ldots,n^2$ is a canonical basis in $\M_n$. In the 
Eq.~\ref{eqn:def-tensor} operation $\Phi$ is applied to the first subsystem 
only.

\todo{to jest do przepisania}
%%%%%%%%%%%%%%%%%%%%%%%%%%%%%%%%%%%%%%%%%%%%%%%%%%%%%%%%%%%%%%%%%%%%%%%%%%%%%%%%
\subsubsection{Partial transposition}
%%%%%%%%%%%%%%%%%%%%%%%%%%%%%%%%%%%%%%%%%%%%%%%%%%%%%%%%%%%%%%%%%%%%%%%%%%%%%%%%
In the case of transposition $M_\Phi$ is equivalent to \SWAP. For example in the
case of $4$-dimensional density matrix
\begin{equation}
X=\left(
\begin{array}{cccc}
 \alpha_{1,1} & \alpha_{1,2} & \alpha_{1,3} & \alpha_{1,4} \\
 \alpha_{2,1} & \alpha_{2,2} & \alpha_{2,3} & \alpha_{2,4} \\
 \alpha_{3,1} & \alpha_{3,2} & \alpha_{3,3} & \alpha_{3,4} \\
 \alpha_{4,1} & \alpha_{4,2} & \alpha_{4,3} & \alpha_{4,4}
\end{array}
\right)
\end{equation}
partial transposition with respect to second subsystems is obtained using
\code{PartialTransposeB[X, 2, 2]} and it gives
\begin{equation}
X^{T_B}=\left(
\begin{array}{cccc}
 \alpha_{1,1} & \alpha_{2,1} & \alpha_{1,3} & \alpha_{2,3} \\
 \alpha_{1,2} & \alpha_{2,2} & \alpha_{1,4} & \alpha_{2,4} \\
 \alpha_{3,1} & \alpha_{4,1} & \alpha_{3,3} & \alpha_{4,3} \\
 \alpha_{3,2} & \alpha_{4,2} & \alpha_{3,4} & \alpha_{4,4}
\end{array}
\right).
\end{equation}

\todo{dodac druga wersje}

%%%%%%%%%%%%%%%%%%%%%%%%%%%%%%%%%%%%%%%%%%%%%%%%%%%%%%%%%%%%%%%%%%%%%%%%%%%%%%%%
\subsubsection{Partial trace}
%%%%%%%%%%%%%%%%%%%%%%%%%%%%%%%%%%%%%%%%%%%%%%%%%%%%%%%%%%%%%%%%%%%%%%%%%%%%%%%%
Operation of tracing out a subsystem can be achieved in a similar manner. We 
define tracing map as
\begin{equation}
\Phi_\mathrm{tr}(\rho) = \tr \rho \Id,
\end{equation}
which for gives
\begin{equation}
D_{\Phi_\mathrm{tr}} =
\left(
\begin{array}{cccc}
 1 & 0 & 0 & 1 \\
 0 & 0 & 0 & 0 \\
 0 & 0 & 0 & 0 \\
 1 & 0 & 0 & 1
\end{array}
\right)
\end{equation}
for one qubit. Note that this matrix after reshuffling is equal to $\Id$ and 
thus $\Phi_\tr$ is CP-TP map.

\todo{dodac druga wersje}

%%%%%%%%%%%%%%%%%%%%%%%%%%%%%%%%%%%%%%%%%%%%%%%%%%%%%%%%%%%%%%%%%%%%%%%%%%%%%%%%
\subsubsection{Partial \SWAP}
%%%%%%%%%%%%%%%%%%%%%%%%%%%%%%%%%%%%%%%%%%%%%%%%%%%%%%%%%%%%%%%%%%%%%%%%%%%%%%%%
Partial transposition and partial trace are the most popular partial operations.

\todo{dodac przyklad}

%%%%%%%%%%%%%%%%%%%%%%%%%%%%%%%%%%%%%%%%%%%%%%%%%%%%%%%%%%%%%%%%%%%%%%%%%%%%%%%%
\section{Examples}\label{sec:examples}
%%%%%%%%%%%%%%%%%%%%%%%%%%%%%%%%%%%%%%%%%%%%%%%%%%%%%%%%%%%%%%%%%%%%%%%%%%%%%%%%

\ldots

%%%%%%%%%%%%%%%%%%%%%%%%%%%%%%%%%%%%%%%%%%%%%%%%%%%%%%%%%%%%%%%%%%%%%%%%%%%%%%%%
\subsection{Reshuffling}
%%%%%%%%%%%%%%%%%%%%%%%%%%%%%%%%%%%%%%%%%%%%%%%%%%%%%%%%%%%%%%%%%%%%%%%%%%%%%%%%

\ldots

%%%%%%%%%%%%%%%%%%%%%%%%%%%%%%%%%%%%%%%%%%%%%%%%%%%%%%%%%%%%%%%%%%%%%%%%%%%%%%%%
\subsection{Operator Schmidt decomposition}
%%%%%%%%%%%%%%%%%%%%%%%%%%%%%%%%%%%%%%%%%%%%%%%%%%%%%%%%%%%%%%%%%%%%%%%%%%%%%%%%

\ldots


%%%%%%%%%%%%%%%%%%%%%%%%%%%%%%%%%%%%%%%%%%%%%%%%%%%%%%%%%%%%%%%%%%%%%%%%%%%%%%%%
\section{Concluding remarks}\label{sec:comclude}
%%%%%%%%%%%%%%%%%%%%%%%%%%%%%%%%%%%%%%%%%%%%%%%%%%%%%%%%%%%%%%%%%%%%%%%%%%%%%%%%
Package \qi\ provide set of functions which aims to simplify the task of
exploring the space of quantum states and understanding quantum operations.

\ldots

%%%%%%%%%%%%%%%%%%%%%%%%%%%%%%%%%%%%%%%%%%%%%%%%%%%%%%%%%%%%%%%%%%%%%%%%%%%%%%%%
\section*{Acknowledgements}
%%%%%%%%%%%%%%%%%%%%%%%%%%%%%%%%%%%%%%%%%%%%%%%%%%%%%%%%%%%%%%%%%%%%%%%%%%%%%%%%
We acknowledge the financial support by Polish Research Network LFPPI.

\appendix
%%%%%%%%%%%%%%%%%%%%%%%%%%%%%%%%%%%%%%%%%%%%%%%%%%%%%%%%%%%%%%%%%%%%%%%%%%%%%%%%
\section{List of provided functions}
%%%%%%%%%%%%%%%%%%%%%%%%%%%%%%%%%%%%%%%%%%%%%%%%%%%%%%%%%%%%%%%%%%%%%%%%%%%%%%%%

\subsection{Kronecker sum and product, symbolic matrix}

\noindent\textbf{$ \text{KroneckerSum} $ }-- KroneckerSum[A,B] returns the Kronecker sum of matrices A and B defined as A$\otimes $1+1$\otimes $B. Alternative syntax A$\oplus $B for KroneckerSum[A,B] is also provided. See also: KroneckerProduct.$  $\\
\noindent\textbf{$ \text{SquareMatrixQ} $ }-- SquareMatrixQ[A] returns True only if A is a square matrix, and gives False otherwise.$  $\\
\noindent\textbf{$ \text{SymbolicMatrix} $ }-- SymbolicMatrix[a,m,n] returns m$\times $n-matrix with elements a[i,j], i=1,...,m, j=1,...,n. If the third argument is ommited this function returns square m$\times $m matrix. This functions can save you some keystrokes and, thanks to TeXForm function, its results can be easily incorporated in LaTeX documents.$  $\\
\noindent\textbf{$ \text{SymbolicVector} $ }-- SymbolicVector[a,n] is equivalent to Matrix[a,n,1] and it returns a vector with m elements a[i],i=1,...,n.$  $\\
\noindent\textbf{$ \text{SymbolicHermitianMatrix} $ }-- SymbolicHermitianMatrix[sym,n] produces a n$\times $n Hermitian matrix. See also: SymbolicMatrix, SymbolicVector.$  $\\
\noindent\textbf{$ \text{ComplexToPoint} $ }-- ComplexToPoint[z] returns a real and an imaginary parts of a complex number z as a pair of real numbers.$  $\\
\noindent\textbf{$ \text{MatrixSqrt} $ }-- MatrixSqrt[A] returns square root for the matrix A.$  $\\
\noindent\textbf{$ \text{MatrixAbs} $ }-- MatrixAbs[A] returns absolute value for matrix A defined as MatrixSqrt[A.A$  ^{\dagger } $]. See also: MatrixSqrt.$  $\\
\noindent\textbf{$ \text{MatrixRe} $ }-- Hermitian part of the matrix A i.e. $ \frac{1}{2}\text{(A+A} ^{\dagger }\text{).} $\\
\noindent\textbf{$ \text{MatrixIm} $ }-- Antyhermitian part of the matrix A i.e. $ \frac{1}{2}\text{(A-A} ^{\dagger }\text{).} $\\
\noindent\textbf{$ \text{ExpectationValue} $ }-- ExpectationValue[s,A] - accepts a state vector or a density matrix as a first argument and calculates the expectation value for the measurement of A in the state s.$  $\\
\noindent\textbf{$ \text{Commutator} $ }-- Commutator[A,B] returns the commutator of matrices A and B i.e. Commutator[A,B] = A.B - B.A.$  $\\
\subsection{Fidelity, trace distance etc.}

\noindent\textbf{$ \text{Fidelity} $ }-- Fidelity[$ \rho _1,\rho _2 $] returns the quantum fidelity between states $ \rho _1 $ and $ \rho _2 $ calculated using a simplified formula as ($\sum $$ \lambda _i)^2 $, where $ \lambda _i $ are the eigenvalues of $ \rho _1\rho _2. $\\
\noindent\textbf{$ \text{Superfidelity} $ }-- Superfidelity[$ \rho _1,\rho _2 $] calculates superfidelity between $ \rho _1 $ and $ \rho _2 $ defined as Tr[$ \rho _1.\rho _2 $] + Sqrt[1-Tr[$ \rho _1.\rho _1 $]]Sqrt[1-Tr[$ \rho _2.\rho _2 $]]. See: J.A. Miszczak et al., Quantum Information $\&$ Computation, Vol.9 No.1$\&$2 (2009).$  $\\
\noindent\textbf{$ \text{Subfidelity} $ }-- Subfidelity[$ \rho _1,\rho _2 $] returns subfidelity between states $ \rho _1 $ and $ \rho _2 $ See: J.A. Miszczak et al., Quantum Information $\&$ Computation, Vol.9 No.1$\&$2 (2009).$  $\\
\noindent\textbf{$ \text{TraceNorm} $ }-- TraceNorm[A] = $\sum $$ \sigma _i $, where $ \sigma _i $ are the singular values of A. See also: TraceDistance.$  $\\
\noindent\textbf{$ \text{TraceDistance} $ }-- TraceDistance[$ \rho _1,\rho _2 $] returns the trace distance between matrices $ \rho _1 $ and $ \rho _2 $, which is defined as $ \frac{1}{2} $tr$|$$ \rho _1-\rho _2\text{$|$.} $\\
\subsection{Commonly used matrices}

\noindent\textbf{$ \text{sx} $ }-- Pauli matrix $ \sigma _y. $\\
\noindent\textbf{$ \text{sy} $ }-- Pauli matrix $ \sigma _y. $\\
\noindent\textbf{$ \text{sz} $ }-- Pauli matrix $ \sigma _z. $\\
\noindent\textbf{$ \text{$\sigma $x} $ }-- Pauli matrix $ \sigma _y $. This is an alternative notation for sx.$  $\\
\noindent\textbf{$ \text{$\sigma $y} $ }-- Pauli matrix $ \sigma _y $. This is an alternative notation for sy.$  $\\
\noindent\textbf{$ \text{$\sigma $z} $ }-- Pauli matrix $ \sigma _z $. This is an alternative notation for sz.$  $\\
\noindent\textbf{$ \text{id} $ }-- Identity matrix for one qubit. See also: IdentityMatrix.$  $\\
\noindent\textbf{$ \text{wh} $ }-- Hadamard gate for one qubit. See also: QFT.$  $\\
\noindent\textbf{$ \text{$\lambda $1} $ }-- Gell-Mann matrix $ \lambda _1. $\\
\noindent\textbf{$ \text{$\lambda $2} $ }-- Gell-Mann matrix $ \lambda _2. $\\
\noindent\textbf{$ \text{$\lambda $3} $ }-- Gell-Mann matrix $ \lambda _3. $\\
\noindent\textbf{$ \text{$\lambda $4} $ }-- Gell-Mann matrix $ \lambda _4. $\\
\noindent\textbf{$ \text{$\lambda $5} $ }-- Gell-Mann matrix $ \lambda _5. $\\
\noindent\textbf{$ \text{$\lambda $6} $ }-- Gell-Mann matrix $ \lambda _6. $\\
\noindent\textbf{$ \text{$\lambda $7} $ }-- Gell-Mann matrix $ \lambda _7. $\\
\noindent\textbf{$ \text{$\lambda $8} $ }-- Gell-Mann matrix $ \lambda _8. $\\
\noindent\textbf{$ \text{Proj} $ }-- Proj[$\{$$ v_1,v_2 $,...,$ v_n $$\}$] returns projectors for the vectors in the input list.$  $\\
\noindent\textbf{$ \text{BaseVectors} $ }-- BaseVectors[n] returns a list with the canonical basis in n-dimensional Hilbert space $ \mathbb{C}^n $. See also: BaseMatrices.$  $\\
\noindent\textbf{$ \text{BaseMatrices} $ }-- BaseMatrices[n] returns a list with the canonical basis in n$\times $n-dimensional Hilbert-Schmidt space $ \mathbb{M}_n $. See also: BaseVectors.$  $\\
\noindent\textbf{$ \text{KroneckerDeltaMatrix} $ }-- KroneckerDeltaMatrix[i,j,d] returns d$\times $d matrix with 1 at position (i,j) and zeros elsewhere.$  $\\
\noindent\textbf{$ \text{Lambda1} $ }-- Lambda1[i,j,n] generalized Pauli matrix. For example Lambda1[1,2,2] is equal to Pauli $\sigma $x. See also: GeneralizedPauliMatrices.$  $\\
\noindent\textbf{$ \text{Lambda2} $ }-- Lambda2[i,j,n] generalized Pauli matrix. For example Lambda2[1,2,2] is equal to $\sigma $y. See also: GeneralizedPauliMatrices.$  $\\
\noindent\textbf{$ \text{Lambda3} $ }-- Lambda3[i,n] generalized Pauli matrix. For example Lambda3[2,2] is equal to $\sigma $z. See also: GeneralizedPauliMatrices.$  $\\
\noindent\textbf{$ \text{GeneralizedPauliMatrices} $ }-- GeneralizedPauliMatrices[n] returns list of generalized Pauli matrices for SU(n). For n=2 these are just Pauli matrices and for n=3 - Gell-Mann matrices. Note that identity matrix is not included in the list. See also: PauliMatrices, GellMannMatrices, $\lambda $, Lambda1, Lambda2, Lambda3.$  $\\
\noindent\textbf{$ \lambda  $ }-- $\lambda $[i,n] is defined as GeneralizedPauliMatrices[n][[i]].$  $\\
\noindent\textbf{$ \text{PauliMatrices} $ }-- Predefined list of Pauli matrices $\{$$ \sigma _x,\sigma _y,\sigma _z $$\}$. Use Map[MatrixForm[$\#$]$\&$,PauliMatrices] to get this list in more readible form.$  $\\
\noindent\textbf{$ \text{GellMannMatrices} $ }-- List of Gell-Mann matrices. Use Map[MatrixForm[$\#$]$\&$,GellMannMatrices] to get this list in more readable form.$  $\\
\noindent\textbf{$ \text{UpperTriangularOnes} $ }-- UpperTriangularOnes[k,dim] returns not normal matirx of dimension dim with 1$<$k$<$dim-1 bands of ones over the diagonal.$  $\\
\noindent\textbf{$ \text{UpperBandOnes} $ }-- UpperBandOnes[k,dim] returns not normal matirx of dimension dim with bands at position k of ones over the diagonal.$  $\\
\subsection{Quantum gates}

\noindent\textbf{$ \text{Swap} $ }-- Swap[d] returns permutation operator $ \underset{i=0}{\overset{n-1}{ \sum }}\underset{j=0}{\overset{n-1}{ \sum }} $$|$i$\rangle \langle $j$|\otimes |$j$\rangle \langle $i$|$ acting on d-dimensional, d=$ n^2 $ space and exchanging two $\surd $d-dimensional subsystems.$  $\\
\noindent\textbf{$ \text{QFT} $ }-- QFT[n,method] - quantum Fourier transform of dimension n. This function accepts second optional argument, which specifies method used in calculation. Parameter method can be equal to 'Symbolic', which is default, or 'Numerical'. The second option makes this function much faster.$  $\\
\noindent\textbf{$ \text{cnot} $ }-- Controlled not matrix for two qubits.$  $\\
\noindent\textbf{$ \text{GeneralizedPauliX} $ }-- Generalized Pauli matrix X. See also: $ \sigma _x $\\
\noindent\textbf{$ \text{GeneralizedPauliZ} $ }-- Generalized Pauli matrix Z. See also: $ \sigma _z $\\
\subsection{Special states}

\noindent\textbf{$ \text{Ket} $ }-- Ket[i,d] returns $|$i$\rangle $ in d-dimensional Hilbert space. See also: StateVector for a different parametrization.$  $\\
\noindent\textbf{$ \text{Ketbra} $ }-- This function can be used in two ways. Ketbra[i,j,d] returns $|$i$\rangle \langle $j$|$ acting on d-dimensional space. See also: Proj. Ketbra[v1,v2] returns the apropritae operator for vectors v1 and v2..$  $\\
\noindent\textbf{$ \text{KetFromDigits} $ }-- KetFromDigits[list,base] - ket vector labeled by a list of digits represented in given base.$  $\\
\noindent\textbf{$ \text{MaxMix} $ }-- MaxMix[n] - the maximally mixed state in a n-dimensional space of density matrices.$  $\\
\noindent\textbf{$ \text{MaxEnt} $ }-- MaxEnt[n] - maximally entangled state in n dimensional vector space. Note that n must be perfect square.$  $\\
\noindent\textbf{$ \text{WernerState} $ }-- WernerState[d,p] - generalized Werner state d$\times $d-dimensional system with the mixing parameter p$\in $[0,1]. This state is defined as p Proj[MaxEnt[d]] + (1-p) MaxMix[d],. See also: MaxEnt, MaxMix..$  $\\
\noindent\textbf{$ \text{IsotropicState} $ }-- IsotropicState[d,p] - isotropic state of dimensions d$\times $d with a parameter p$\in $[0,1]. This state is defined as p Proj[MaxEnt[d]] + (1-p)/($ d^2 $-1)(I-Proj[MaxEnt[d]]]). This family of states is invariant for the operation of the form U$\otimes $$ U^*. $\\
\subsection{Schmidt decomposition}

\noindent\textbf{$ \text{VectorSchmidtDecomposition} $ }-- VectorSchmidtDecomposition[vec,d1,d2] - Schmidt decomposition of the vector vec in d1$\times $d2-dimensional Hilbert space.$  $\\
\noindent\textbf{$ \text{OperatorSchmidtDecomposition} $ }-- OperatorSchmidtDecomposition[mtx,d1,d2] - Schmidt decomposition of mtx in the Hilbert-Schmidt space of matrices of dimension d1$\times $d2.$  $\\
\noindent\textbf{$ \text{SchmidtDecomposition} $ }-- SchmidtDecomposition[e,d1,d2] - accepts a vector or a matrix as a first argument and returns apropriate Schmidt decomposition. See also: VectorSchmidtDecomposition, OperatorSchmidtDecomposition.$  $\\
\subsection{Reshaping, vectorization and reshuffling}

\noindent\textbf{$ \text{Vec} $ }-- Vec[A] - vectorization of the matrix A column by column. See also: Res.$  $\\
\noindent\textbf{$ \text{Unvec} $ }-- Unvec[v,c] - de-vectorization of the vector into the matrix with c columns. If the second parameter is omitted then it is assumed that v can be mapped into square matrix. See also: Unres, Vec.$  $\\
\noindent\textbf{$ \text{Res} $ }-- Res[A] is equivalent to Vec[Transpose[A]]. Reshaping maps matrix A into a vector row by row.$  $\\
\noindent\textbf{$ \text{Unres} $ }-- Unres[v,c] - de-reshaping of the vector into a matrix with c columns. If the second parameter is omitted then it is assumed that v can be mapped into a square matrix. See also: Unvec, Res.$  $\\
\noindent\textbf{$ \text{Reshuffle} $ }-- Reshuffle[$\rho $,method] for square matrix of dimensions $ d^2\times d^2 $, where d is an Integer, returns reshuffled matrix, available methods: $\texttt{"$ $Fast$\texttt{"}$ (default) and $\texttt{"}$BaseMatrices$\texttt{"}$ (slow). Use ReshuffleGeneral for matrices with dimensions different then }d^2\times d^2 $, where d is an Integer. See also: ReshuffleGeneral, ReshuffleBase, Reshuffle2.$  $\\
\noindent\textbf{$ \text{Reshuffle2} $ }-- Reshuffle2[$\rho $,method] for square matrix of dimensions $ d^2\times d^2 $, where d is an Integer, returns reshuffled matrix given by alternative definition of the reshuffling operation, available methods: $\texttt{"$ $Fast$\texttt{"}$ (default) and $\texttt{"}$BaseMatrices$\texttt{"}$ (slow). Use ReshuffleGeneral2 for matrices with dimensions different then }d^2\times d^2 $, where d is an Integer. See also: ReshuffleGeneral2, ReshuffleBase2, Reshuffle.$  $\\
\noindent\textbf{$ \text{ReshuffleBase} $ }-- ReshuffleBase[$\rho $,m,n] returns representation of the m$\times $n-dimensional square matrix $\rho $ in the basis consisting of product matrices. If  the matrix $\rho $ has dimension $ d^2 $ then two last arguments can be omitted. In this case one obtains a reshuffle in the basis constructed by using two bases of d-dimensional Hilbert-Schmidt matrix spaces. See also: Reshuffle, ReshuffleGeneral, Reshuffle2.$  $\\
\noindent\textbf{$ \text{ReshuffleBase2} $ }-- Alternative definition of the reshuffling operation. Reshuffle2[$\rho $,m,n] returns representation of the m$\times $n-dimensional square matrix $\rho $ in the basis consisting of product matrices which are transposed versions of standard base matrices. If the matrix $\rho $ has dimension $ d^2 $ then two last arguments can be omitted. In this case one obtains a reshuffle in the basis constructed by using two bases of d-dimensional Hilbert-Schmidt matrix spaces. See: See also: Reshuffle2, ReshuffleGeneral, Reshuffle, BaseMatrices$  $\\
\noindent\textbf{$ \text{ReshuffleGeneral} $ }-- ReshuffleGeneral[$\rho $,n1,m1,n2,m2] for matrix of size (n1 n2)$\times $(m1 m2) returns a reshuffled matrix.$  $\\
\noindent\textbf{$ \text{ReshuffleGeneral2} $ }-- ReshuffleGeneral2[$\rho $,n1,m1,n2,m2] for matrix of size (n1 n2)$\times $(m1 m2) returns a reshuffled matrix - given by alternative definition of the reshuffling operation.$  $\\
\noindent\textbf{$ \text{MatrixElement} $ }-- MatrixElement[n,$\nu $,m,$\mu $,dim,A] - returns the matrix element of a matrix A indexed by two double indices n, $\nu $ and m, $\mu $ of the composite sytem of dimensions dim=dim1*dim2.$  $\\
\noindent\textbf{$ \text{ReshufflePermutation} $ }-- ReshufflePermutation[dim1,dim2] - permutation matrix equivalent to the reshuffling operation on dim1$\times $dim2-dimensional system. See also: Reshuffle.$  $\\
\noindent\textbf{$ \text{ReshufflePermutation2} $ }-- ReshufflePermutation2[dim1,dim2] - permutation matrix equivalent to the alternative reshuffling operation on dim1$\times $dim2-dimensional system. See also: Reshuffle.$  $\\
\noindent\textbf{$ \text{ProductSuperoperator} $ }-- ProductSuperoperator[$\Psi $,$\Phi $] computes a product superoperator of superoperatos $\Psi $ and $\Phi $.$  $\\
\subsection{Parametrizations}

\noindent\textbf{$ \text{Unitary2} $ }-- Unitary2[$\alpha $,$\beta $,$\gamma $,$\delta $] returns the Euler parametrization of U(2).$  $\\
\noindent\textbf{$ \text{SpecialUnitary2} $ }-- SpecialUnitary2[$\beta $,$\gamma $,$\delta $] returns the Euler parametrization of SU(2). This is equivalent to Unitary2[0,$\beta $,$\gamma $,$\delta $].$  $\\
\noindent\textbf{$ \text{Unitary3} $ }-- Unitary3[$\alpha $,$\beta $,$\gamma $,$\tau $,a,b,c,ph] returns the Euler parametrization of U(3).$  $\\
\noindent\textbf{$ \text{Unitary4Canonical} $ }-- Parametrization of non-local unitary matrices for two qubits. See: B. Kraus, J.I. Cirac, Phys. Rev. A 63, 062309 (2001), quant-ph/0011050v1.$  $\\
\noindent\textbf{$ \text{ProbablityVector} $ }-- ProbablityVector[$\{$$ \theta _1 $,...,$ \theta _n $$\}$] returns probability vectors of dimension n+1 parametrize with $\{$$ \theta _1 $,...,$ \theta _n $$\}$. See also: StateVector.$  $\\
\noindent\textbf{$ \text{StateVector} $ }-- StateVector[$\{$$ \theta _1 $,...,$ \theta _n,\phi _{n+1} $,...,$ \phi _{2 n} $$\}$] returns pure n+1-dimensional pure state (ket vector) constructed form probability distribution parametrize by numbers $\{$$ \theta _1 $,...,$ \theta _n $$\}$ and phases $\{$$ \phi _1 $,...,$ \phi _n $$\}$. See also: ProbablityVector, SymbolicVector.$  $\\
\subsection{One-qubit states}

\noindent\textbf{$ \text{QubitKet} $ }-- QubitKet[$\alpha $,$\beta $] parametrization of the pure state (as a state vector) for one qubit as (Cos[$\alpha $] Exp[i$\beta $], Sin[$\alpha $]). This is equivalent to StateVector[$\{\alpha $,$\beta \}$]. See also: QubitPureState, StateVector.$  $\\
\noindent\textbf{$ \text{QubitPureState} $ }-- QubitPureState[$\alpha $,$\beta $] - a parametrization of the pure state as a density matrix for one qubit. This is just a alias for Proj[QubitKet[$\alpha $,$\beta $]]. See also: QubitKet.$  $\\
\noindent\textbf{$ \text{QubitBlochState} $ }-- QubitBlochState[$\rho $] - a parametrization of the one-qubit mixed state on the Bloch sphere.$  $\\
\noindent\textbf{$ \text{QubitGeneralState} $ }-- QubitGeneralState[$\alpha $,$\beta $,$\gamma $,$\delta $,$\lambda $] - Parametrization of the one-qubit mixed state using rotations and eigenvalues. Returns one-qubits density matrix with eigenvalues $\lambda $ and 1-$\lambda $ rotated as U.diag($\lambda $,1-$\lambda $).$ U^{\dagger } $ with U defined by parameters $\alpha $,$\beta $,$\gamma $ and $\delta $.$  $\\
\subsection{Quantum channels}

\noindent\textbf{$ \text{IdentityChannel} $ }-- IdentityChannel[n,$\rho $] - apply the identity operation to a n-dimensional density matrix $\rho $.$  $\\
\noindent\textbf{$ \text{TransposeChannel} $ }-- TransposeChannel[n,$\rho $] - apply the transposition operation to a n-dimensional density matrix $\rho $. Note that this operations is not completely positive.$  $\\
\noindent\textbf{$ \text{DepolarizingChannel} $ }-- DepolarizingChannel[n,p,$\rho $] - apply the completely depolarizing channel with parameter p acting to a n-dimensional input state $\rho $. See also: QubitDepolarizingKraus, HolevoWernerChannel.$  $\\
\noindent\textbf{$ \text{HolevoWernerChannel} $ }-- HolevoWernerChannel[n,p,$\rho $] - apply the Holeve-Werner channel, also known as transpose-depolarizing channel, with parameter p acting to a n-dimensional input state $\rho $. See also: DepolarizingChannel.$  $\\
\noindent\textbf{$ \text{ChannelToMatrix} $ }-- ChannelToMatrix[E,d] returns matrix representation of a channel E acting on d-dimensional state space. First argument should be a pure function E such that E[$\rho $] transforms input state according to the channel definition.$  $\\
\noindent\textbf{$ \text{GeneralizedPauliKraus} $ }-- GeneralizedPauliKraus[d,P] - list of Kraus operators for d-dimensional generalized Pauli channel with the d-dimesnional matrix of parameters P. See: M. Hayashi, Quantum Information An Introduction, Springer 2006, Example 5.8, p. 126.$  $\\
\noindent\textbf{$ \text{ApplyKraus} $ }-- ApplyKraus[ck,$\rho $] - apply channel ck, given as a list of Kraus operators, to the input state $\rho $. See also: ApplyUnitary, ApplyChannel.$  $\\
\noindent\textbf{$ \text{ApplyUnitary} $ }-- ApplyUnitary[U,$\rho $] - apply unitary a unitary matrix U to the input state $\rho $. See also: ApplyKraus, ApplyChannel.$  $\\
\noindent\textbf{$ \text{ApplyChannel} $ }-- ApplayChannel[f,$\rho $] - apply channel f, given as a pure function, to the input state $\rho $. See also: ApplyUnitary, ApplyKraus.$  $\\
\noindent\textbf{$ \text{Superoperator} $ }-- Superoperator[kl] returns matrix representation of quantum channel given as a list of Kraus operators. Superoperator[fun,dim] is just am alternative name for ChannelToMatrix[fun,dim] and returns matrix representation of quantum channel, given as a pure function, acting on dim-dimensional space. So Superoperator[DepolarizingChannel[2,p,$\#$]$\&$,2] and Superoperator[QubitDepolarizingKraus[p]] returns the same matrix. See also: ChannelToMatrix.$  $\\
\noindent\textbf{$ \text{DynamicalMatrix} $ }-- Dynamical matrix of quantum channel given as a list of Kraus operators (DynamicalMatrix[ch]) or as a function fun action on dim-dimensional space (DynamicalMatrix[fun,dim]). See also: Superoperator, ChannelToMatrix.$  $\\
\noindent\textbf{$ \text{Jamiolkowski} $ }-- Jamiolkowski[K] gives the image of the Jamiolkowski isomorphism for the channel given as the list of Karus operators K. Jamiolkowski[fun,dim] gives the image of the Jamiolkowski isomorphism for the channel given as a function fun action on dim-dimensional space. See also: Superoperator, ChannelToMatrix, DynamicalMatrix.$  $\\
\noindent\textbf{$ \text{TPChannelQ} $ }-- Performs some checks on Kraus operators. Use this if you want to check if they represent quantum channel.$  $\\
\noindent\textbf{$ \text{ExtendKraus} $ }-- ExtendKraus[ch,n] - produces n-fold tensor products of Kraus operators from the list ch.$  $\\
\noindent\textbf{$ \text{SuperoperatorToKraus} $ }-- Finds Kraus operators for a given super operator$  $\\
\subsection{Partial trace and transposition}

\noindent\textbf{$ \text{PartialTransposeA} $ }-- PartialTransposeA[$\rho $,m,n] performs partial transposition on the m-dimensional (first) subsystem of the m$\times $n-state.$  $\\
\noindent\textbf{$ \text{PartialTransposeB} $ }-- PartialTransposeB[$\rho $,m,n] performs partial transposition on the n-dimensional (second) subsystem of the m$\times $n-state.$  $\\
\noindent\textbf{$ \text{PartialTraceA} $ }-- PartialTraceA[$\rho $,m,n] performs partial trace on m$\times $n-dimensional density matrix $\rho $ with respect to the m-demensional (first) subsystem. This function is implemented using composition of channels. Use PartialTraceGeneral for better performance.$  $\\
\noindent\textbf{$ \text{PartialTraceB} $ }-- PartialTraceB[$\rho $,m,n] performs partial trace on m$\times $n-dimensional density matrix $\rho $ with respect to the n-dimensional (second) subsystem. This function is implemented using composition of channels. Use PartialTraceGeneral for better performance.$  $\\
\noindent\textbf{$ \text{PartialTraceGeneral} $ }-- PartialTraceGeneral[$\rho $,dim,sys] - Returns the partial trace, according to system sys, of density matrix $\rho $ composed of subsystems of dimensions dim=$\{$dimA, dimB$\}$. See also: PartialTraceA, PartialTraceB.$  $\\
\noindent\textbf{$ \text{PartialTransposeGeneral} $ }-- PartialTransposeGeneral[$\rho $,dim,sys] - Returns the partial transpose, according to system sys, of density matrix $\rho $ composed of subsystems of dimensions dim=$\{$dimA,dimB$\}$. $  $\\
\subsection{Entanglement}

\noindent\textbf{$ \text{Concurrence4} $ }-- Concurrence4[$\rho $] returns quantum concurrence of a density matrix $\rho $ representing a state of two-qubit system. This function uses Chop to provide numerical results.$  $\\
\noindent\textbf{$ \text{Negativity} $ }-- Negativity[$\rho $,m,n] returns the sum of negative eigenvalues of the density matrix $\rho \in $$ \mathbb{M}_{m\times n} $ after their partial transposition with respect to the first subsystem.$  $\\
\subsection{One-qubit quantum channels}

\noindent\textbf{$ \text{QubitBitflipChannel} $ }-- BitflipChannel[p,$\rho $] applies bif-flip channel to the input state $\rho $. See also: QubitBitflipKraus.$  $\\
\noindent\textbf{$ \text{QubitPhaseflipChannel} $ }-- QubitPhaseflipChannel[p,$\rho $] applies phase-flip channel to the input state $\rho $. See also: QubitPhaseflipKraus.$  $\\
\noindent\textbf{$ \text{QubitBitphaseflipChannel} $ }-- QubitBitphaseflipChannel[p,$\rho $] applies bit-phase-flip channel to the input state $\rho $. See also: QubitPhaseflipKraus.$  $\\
\noindent\textbf{$ \text{QubitDepolarizingKraus} $ }-- Kraus operators of the depolarizing channel for one qubit. Note that it gives maximally mixed state for p=1.$  $\\
\noindent\textbf{$ \text{QubitDecayKraus} $ }-- Kraus operators of the decay channel, also know as amplitude damping, for one qubit.$  $\\
\noindent\textbf{$ \text{QubitPhaseKraus} $ }-- Kraus operators for one qubit phase damping channel.$  $\\
\noindent\textbf{$ \text{QubitBitflipKraus} $ }-- Kraus operators for one qubit bit-flip channel.$  $\\
\noindent\textbf{$ \text{QubitPhaseflipKraus} $ }-- Kraus operators for one qubit phase-flip channel.$  $\\
\noindent\textbf{$ \text{QubitBitphaseflipKraus} $ }-- Kraus operators for one qubit bit-phase-flip channel.$  $\\
\noindent\textbf{$ \text{QubitDynamicalMatrix} $ }-- QubitDynamicalMatrix[$ \kappa _x,\kappa _y,\kappa _z,\eta _x,\eta _y,\eta _z $] returns parametrization of one-qubit dynamical matrix. See: I. Bengtsson, K. Zyczkowski, Geometry of Quantum States, Chapter 10, Eg.(10.81).$  $\\
\noindent\textbf{$ \text{QubitDaviesSuperoperator} $ }-- QubitDaviesSuperoperator[a,c,p] returns a superoperator matrix for one-qubit Davies channel with parameters a and c and the stationary state (p,1-p).$  $\\
\subsection{One-qutrit channels}

\noindent\textbf{$ \text{QutritSpontaneousEmissionKraus} $ }-- QutritSpontaneousEmissionKraus[A1,A2,t] Kraus operators for qutrit spontaneous emission channel with parameters A1, A2, t $>$= 0. See: A. Checinska, K. Wodkiewicz, Noisy Qutrit Channels, arXiv:quant-ph/0610127v2.$  $\\
\subsection{Entropy}

\noindent\textbf{$ \text{Log0} $ }-- Log0[x] is equal to Log[2,x] for x$>$0 and 1 for x=0.$  $\\
\noindent\textbf{$ \eta  $ }-- $\eta $[x] = -x Log[2,x].$  $\\
\noindent\textbf{$ \text{$\eta $2} $ }-- $\eta $2[x] = $\eta $[x]+$\eta $[1-x].$  $\\
\noindent\textbf{$ \text{QuantumEntropy} $ }-- QuantumEntropy[m] - von Neuman entropy for the matrix m.$  $\\
\noindent\textbf{$ \text{QuantumChannelEntropy} $ }-- QuantumChannelEntropy[ch] - von Neuman entropy of the quantum channel calculated as a von Neuman entropy for the image of this channel in Jamiolkowski isomorphism. See also: Jamiolkowski, Superoperator.$  $\\
\subsection{Distribution of eigenvalues}

\noindent\textbf{$ \delta  $ }-- $\delta $[a] is equivalent to $\delta $[x,''Dirac''] and it represents Dirac delta at x. If the second argument is ''Indicator'', $\delta $[x,''Indicator''] is equivalent to DiscreteDelta[x].$  $\\
\noindent\textbf{$ \text{VandermondeMatrix} $ }-- VandermondeMatrix[$\{$$ x_1\text{,...}x_n $$\}$] - Vandermonde matrix for variables ($ x_1 $,...,$ x_n\text{).} $\\
\noindent\textbf{$ \text{ProdSum} $ }-- ProdSum[$\{$$ x_1 $,...,$ x_n $$\}$] gives $ \prod _{i<j}^nx_i+x_j. $\\
\noindent\textbf{$ \text{ProdDiff2} $ }-- ProdDiff2[$\{$$ x_1 $,...,$ x_n $$\}$] is equivalent to Det[VandermondeMatrix[$\{$$ x_1 $,...,$ x_n $$\}$]$ ]^2 $ and gives a discriminant of the polynomial with roots $\{$$ x_1 $,...,$ x_n $$\}$.$  $\\
\noindent\textbf{$ \text{ProbBuresNorm} $ }-- ProbBNorm[n] - Normalization factor used for calculating probability distribution of eigenvalues of matrix of dimension N according to Bures distance.$  $\\
\noindent\textbf{$ \text{ProbBures} $ }-- ProbBures[$\{$$ x_1\text{,...}x_n $$\}$,$\delta $] - Joint probability distribution of eigenvalues $\lambda $ = $\{$$ x_1\text{,...}x_n $$\}$of a matrix according to Bures distance. By default $\delta $ is assumed to be Dirac delta. Other possible values: ''Indicator''$  $\\
\noindent\textbf{$ \text{ProbHSNorm} $ }-- Normalization factor used for calculating probability distribution of eigenvalues of matrix of dimension N according to Hilbert-Schmidt distribution.$  $\\
\noindent\textbf{$ \text{ProbHS} $ }-- ProbHS[$\{$$ x_1\text{,...}x_n $$\}$,$\delta $] Probability distribution of eigenvalues $\lambda $ = $\{$$ x_1\text{,...}x_n $$\}$ of a matrix according to Hilbert-Schmidt distance. By default $\delta $ is assumed to be Dirac delta. Other possible values: ''Indicator''$  $\\
\subsection{Random states and operations}

\noindent\textbf{$ \text{RandomSimplex} $ }-- RandomSimplex[d,$\alpha $] generates a point on a d-dimensional simplex according to the Dirichlet distibution with parameter $\alpha $.$\backslash $n RandomSimplex[d] uses the algorithm from the book 'Luc Devroye, Non-Uniform Random Variate Generation, Chapter 11, p. 568' and gives the flat distribution.$  $\\
\noindent\textbf{$ \text{RandomKet} $ }-- RandomKet[d] - random ket in d-dimensional space. See: T. Radtke, S. Fritzsche, Comp. Phys. Comm., Vol. 179, No. 9, p. 647-664.$  $\\
\noindent\textbf{$ \text{RandomProductKet} $ }-- RandomProductKet[$\{$dim1,dim2,...,dimN$\}$] - random pure state (ket vector) of the tensor product form with dimensions of subspaces specified dim1, dim2,...,dimN.$  $\\
\noindent\textbf{$ \text{RandomNormalMatrix} $ }-- RandomNormalMatrix[d] - random normal matrix of dimension d.$  $\\
\noindent\textbf{$ \text{RandomDynamicalMatrix} $ }-- RandomDynamicalMatrix[d,k] returns dynamical matrix of operation acting on d-dimensional states with k eigenvalues equal to 0. Thanks to Wojtek Bruzda.$  $\\
\noindent\textbf{$ \text{GinibreMatrix} $ }-- GinibreMatrix[m,n] returns complex matrix of dimension m$\times $n with normal distribution of real and imaginary parts.$  $\\
\noindent\textbf{$ \text{RandomProductNumericalRange} $ }-- RandomLocalNumericalRange[M,$\{$dim1,dim2,...,dimN$\}$,n] returns n points from the product numerical range of the matrix M with respect to division specified as $\{$dim1,dim2,...,dimN$\}$. Note that dim1$\times $dim2$\times $...$\times $dimN must be equal to the dimension of matrix M.$  $\\
\noindent\textbf{$ \text{RandomMaximallyEntangledNumericalRange} $ }-- RandomMaximallyEntangledNumericalRange[M,n] returns n points from the maximally entangled numerical range of the matrix M with respect to division Sqrt[dim[M]]$\times $Sqrt[dim[M]].$  $\\
\noindent\textbf{$ \text{RandomSpecialUnitary} $ }-- Random special unitary matrix. Thanks to Rafal Demkowicz-Dobrzanski.$  $\\
\noindent\textbf{$ \text{RandomUnitaryEuler} $ }-- Random unitary matrix. Thanks to Rafal Demkowicz-Dobrzanski.$  $\\
\noindent\textbf{$ \text{RandomUnitaryQR} $ }-- Random unitary matrix using QR decomposition. F. Mezzadri, See: NOTICES of the AMS, Vol. 54 (2007), 592-604$  $\\
\noindent\textbf{$ \text{RandomUnitary} $ }-- Random unitary matrix. This function can be used as RandomUnitary[dim,$\texttt{"$ $QR$\texttt{"}$] (default and faster) or RandomUnitary[dim,$\texttt{"}$Euler$\texttt{"}$] (slower) the first argument is the dimensions and the second argument specifies the generation method. See also RandomUnitaryEuler and RandomUnitaryQR.} $\\
\noindent\textbf{$ \text{RandomState} $ }-- RandomState[d,dist] - random density matrix of dimension d. Argument dist can be ''HS'' (default value) or ''Bures'' or an integer K. ''HS'' gives uniform distribution with respect to the Hilbert-Schmidt measure. ''Bures'' gives random state distributed according to Bures measure. If dist is given as an integer K, the state is generated with respect to induced measure with an ancilla system od dimension K.$  $\\
\subsection{Random vectors}

\noindent\textbf{$ \text{RandomComplexUnitVector} $ }-- RandomComplexUnitVector[n] returns a normalized, n-dimensional vector of complex numbers.$  $\\
\noindent\textbf{$ \text{RandomRealUnitVector} $ }-- RandomRealUnitVector[n] returns a normalized, n-dimensional vector of real numbers$  $\\
\noindent\textbf{$ \text{RandomUnitVector} $ }-- RandomUnitVector[n] returns a normalized, n-dimensional vector of complex numbers. If the second argument is set to 'Real', thef unction will output a vector over $ \mathbb{R}^n $. See also: RandomKet.$  $\\
\noindent\textbf{$ \text{RandomEntangledUnitVector} $ }-- RandomEntangledUnitVector[n] returns a maximally entangled unit vector on the n-dimensional vector space.$  $\\
\noindent\textbf{$ \text{RandomUnitVectorSchmidt} $ }-- RandomUnitVectorSchmidt[n,r] returns a unit vector on n-dimensional space with a Schmidt rank r. Note that r has to smaller or equal $\surd $n and n has to be a perfect square.$  $\\
\subsection{Numerical range}

\noindent\textbf{$ \text{NumericalRangeBound} $ }-- NumericalRangeBound[A,dx] - bound of numerical range of matrix A calculated with given step dx. Default value of dx is 0.01. Ref: Carl C. Cowen, Elad Harel, An Effective Algorithm for Computing the Numerical Range. Technical report, Dep. of Math. Purdue University, 1995.$  $\\
\subsection{Bloch Representation}

\noindent\textbf{$ \text{BlochVector} $ }-- BlochVector[A] - for a square matrix A returns a vector of coefficients obtained from expansion on normed generalized Pauli matrices. See also: GeneralizedPauliMatrices.$  $\\
\noindent\textbf{$ \text{StateFromBlochVector} $ }-- StateFromBlochVector[v] - returns a matrix of appropriate dimension from Bloch vector, i.e. coefficients treated as coefficients from expansion on normalized generalized Pauli matrices. See also: GeneralizedPauliMatrices.$  $\\


\bibliography{qi}
\bibliographystyle{elsarticle-num}

\end{document}
\endinput
%%
%% End of file `elsarticle-template-num.tex'.
