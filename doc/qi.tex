%%
%% This is file `elsarticle-template-num.tex',
%% generated with the docstrip utility.
%%
%% The original source files were:
%%
%% elsarticle.dtx  (with options: `numtemplate')
%% 
%% Copyright 2007, 2008 Elsevier Ltd.
%% 
%% This file is part of the 'Elsarticle Bundle'.
%% -------------------------------------------
%% 
%% It may be distributed under the conditions of the LaTeX Project Public
%% License, either version 1.2 of this license or (at your option) any
%% later version.  The latest version of this license is in
%%    http://www.latex-project.org/lppl.txt
%% and version 1.2 or later is part of all distributions of LaTeX
%% version 1999/12/01 or later.
%% 
%% The list of all files belonging to the 'Elsarticle Bundle' is
%% given in the file `manifest.txt'.
%% 

%% Template article for Elsevier's document class `elsarticle'
%% with numbered style bibliographic references
%% SP 2008/03/01

\documentclass[final,5p,times,twocolumn]{elsarticle}

%% Use the option review to obtain double line spacing
%% \documentclass[authoryear,preprint,review,12pt]{elsarticle}

%% Use the options 1p,twocolumn; 3p; 3p,twocolumn; 5p; or 5p,twocolumn
%% for a journal layout:
%% \documentclass[final,1p,times]{elsarticle}
%% \documentclass[final,1p,times,twocolumn]{elsarticle}
%% \documentclass[final,3p,times]{elsarticle}
%% \documentclass[final,3p,times,twocolumn]{elsarticle}
%% \documentclass[final,5p,times]{elsarticle}
%% \documentclass[final,5p,times,twocolumn]{elsarticle}

%% if you use PostScript figures in your article
%% use the graphics package for simple commands
%% \usepackage{graphics}
%% or use the graphicx package for more complicated commands
%% \usepackage{graphicx}
%% or use the epsfig package if you prefer to use the old commands
%% \usepackage{epsfig}

%% The amssymb package provides various useful mathematical symbols
\usepackage{amssymb}
%% The amsthm package provides extended theorem environments
%% \usepackage{amsthm}

%% The lineno packages adds line numbers. Start line numbering with
%% \begin{linenumbers}, end it with \end{linenumbers}. Or switch it on
%% for the whole article with \linenumbers.
%% \usepackage{lineno}

\newcommand{\Mathematica}{\emph{Mathematica}}
\newcommand{\wek}{\mathbf{vec}}
\newcommand{\res}{\mathbf{res}}
\newcommand{\1}{{\rm 1\hspace{-0.9mm}l}}
\newcommand{\Id}{\1}
\newcommand{\SWAP}{\ensuremath{\mathrm{SWAP}}}
\newcommand{\tr}{\mathrm{tr}}
\newcommand{\M}{\ensuremath{\mathbb{M}}}

\journal{Computer Physics Communications}

\begin{document}

\begin{frontmatter}

%% Title, authors and addresses

%% use the tnoteref command within \title for footnotes;
%% use the tnotetext command for theassociated footnote;
%% use the fnref command within \author or \address for footnotes;
%% use the fntext command for theassociated footnote;
%% use the corref command within \author for corresponding author footnotes;
%% use the cortext command for theassociated footnote;
%% use the ead command for the email address,
%% and the form \ead[url] for the home page:
%% \title{Title\tnoteref{label1}}
%% \tnotetext[label1]{}
%% \author{Name\corref{cor1}\fnref{label2}}
%% \ead{email address}
%% \ead[url]{home page}
%% \fntext[label2]{}
%% \cortext[cor1]{}
%% \address{Address\fnref{label3}}
%% \fntext[label3]{}

\title{Q1---A \Mathematica\ package for analyzing geometry of quantum states}

%% use optional labels to link authors explicitly to addresses:
%% \author[label1,label2]{}
%% \address[label1]{}
%% \address[label2]{}

\author{P.~Gawron}
\author{J.~A.~Miszczak\corref{cor1}}
\author{Z.~Pucha{\l}a}
\address{Institute of Theoretical and Applied Informatics, Polish Academy of 
Sciences, Ba{\l}tycka 5, 44-100 Gliwice, Poland}

\begin{abstract}
We present a package of functions for \Mathematica\ computer algebra system, 
which implements number of functions used in the analysis of quantum states.

In contrast to many available packages for symbolic and numerical simulation of
quantum computation presented package is focused on geometrical aspects of 
quantum information theory.

\end{abstract}

\begin{keyword}
%% keywords here, in the form: keyword \sep keyword
quantum information\sep symbolic computation
%% PACS codes here, in the form: \PACS code \sep code
\PACS

%% MSC codes here, in the form: \MSC code \sep code
%% or \MSC[2008] code \sep code (2000 is the default)

\end{keyword}

\end{frontmatter}

%% \linenumbers

%% main text
%%%%%%%%%%%%%%%%%%%%%%%%%%%%%%%%%%%%%%%%%%%%%%%%%%%%%%%%%%%%%%%%%%%%%%%%%%%%%%%%
\section{Introduction}\label{sec:intro}
%%%%%%%%%%%%%%%%%%%%%%%%%%%%%%%%%%%%%%%%%%%%%%%%%%%%%%%%%%%%%%%%%%%%%%%%%%%%%%%%

Developement of quantum information processing demands computational tools 
for analyzing starnge behaviour of quantum mechanical states and operations \cite{qdensity,radtke08simulation}.

For a most comprehensive list of quantum computing simulators see \cite{quantiki}.

%% The Appendices part is started with the command \appendix;
%% appendix sections are then done as normal sections
%% \appendix

%%%%%%%%%%%%%%%%%%%%%%%%%%%%%%%%%%%%%%%%%%%%%%%%%%%%%%%%%%%%%%%%%%%%%%%%%%%%%%%%
\subsection{Overivew of functionality}
%%%%%%%%%%%%%%%%%%%%%%%%%%%%%%%%%%%%%%%%%%%%%%%%%%%%%%%%%%%%%%%%%%%%%%%%%%%%%%%%


%%%%%%%%%%%%%%%%%%%%%%%%%%%%%%%%%%%%%%%%%%%%%%%%%%%%%%%%%%%%%%%%%%%%%%%%%%%%%%%%
\subsection{Desing principles}
%%%%%%%%%%%%%%%%%%%%%%%%%%%%%%%%%%%%%%%%%%%%%%%%%%%%%%%%%%%%%%%%%%%%%%%%%%%%%%%%

\begin{enumerate}
\item user know what she/he is doing -- function do not expect to get positive/unitary/\ldots\
matrix
\item only basic \Mathematica\ structures are used -- functions operator on 
plain \Mathematica\ lists and return lists
\item as simple as possible -- functions have mimimum reasonable numer of arguments
\item only basic version are provided -- functions perform only simple 
operations, for example in order to get partial trace on few subsystems user
needs to perform premutation and then applay partial trace
\item some data are used more often thn other -- Q1 predefines some commomly used
matrice, for example matrice for \SWAP\ operatio for some small dimensions.
\end{enumerate}
%%%%%%%%%%%%%%%%%%%%%%%%%%%%%%%%%%%%%%%%%%%%%%%%%%%%%%%%%%%%%%%%%%%%%%%%%%%%%%%%
\section{Parametrizations of states}\label{sec:states}
%%%%%%%%%%%%%%%%%%%%%%%%%%%%%%%%%%%%%%%%%%%%%%%%%%%%%%%%%%%%%%%%%%%%%%%%%%%%%%%%

%%%%%%%%%%%%%%%%%%%%%%%%%%%%%%%%%%%%%%%%%%%%%%%%%%%%%%%%%%%%%%%%%%%%%%%%%%%%%%%%
\subsection{One-qubit states}
%%%%%%%%%%%%%%%%%%%%%%%%%%%%%%%%%%%%%%%%%%%%%%%%%%%%%%%%%%%%%%%%%%%%%%%%%%%%%%%%

%%%%%%%%%%%%%%%%%%%%%%%%%%%%%%%%%%%%%%%%%%%%%%%%%%%%%%%%%%%%%%%%%%%%%%%%%%%%%%%%
\subsection{Generalizations}
%%%%%%%%%%%%%%%%%%%%%%%%%%%%%%%%%%%%%%%%%%%%%%%%%%%%%%%%%%%%%%%%%%%%%%%%%%%%%%%%
\cite{tilma02generalized}

%%%%%%%%%%%%%%%%%%%%%%%%%%%%%%%%%%%%%%%%%%%%%%%%%%%%%%%%%%%%%%%%%%%%%%%%%%%%%%%%
\section{Partial operation}
%%%%%%%%%%%%%%%%%%%%%%%%%%%%%%%%%%%%%%%%%%%%%%%%%%%%%%%%%%%%%%%%%%%%%%%%%%%%%%%%

%%%%%%%%%%%%%%%%%%%%%%%%%%%%%%%%%%%%%%%%%%%%%%%%%%%%%%%%%%%%%%%%%%%%%%%%%%%%%%%%
\subsection{Partial transposition}
%%%%%%%%%%%%%%%%%%%%%%%%%%%%%%%%%%%%%%%%%%%%%%%%%%%%%%%%%%%%%%%%%%%%%%%%%%%%%%%%
Q1 implements partial operations using general method for constructing channels
acting on subsystems. This method is based on formula
\begin{equation}\label{eqn:def-tensor}
(\Phi\otimes\Id)(\rho) = 
\left(\res^{-1}\left(M_\Phi\res\left(\rho^R\right)\right)\right)^R
\end{equation}
where ${}^R$ denotes the reshuffling operation and $M_\Phi$ denotes the matrix
of the linear map $\Phi$
\begin{equation}
M_\Phi = \tr \epsilon_i \Phi(\epsilon_j),
\end{equation}
where $\{\epsilon_i\}_i=1,\ldots,n^2$ is a canonical basis in $\M_n$. In the 
Eq.~\ref{eqn:def-tensor} operation $\Phi$ is applied to the first subsystem 
only.

In the case of transposition $M_\Phi$ is equivalent to \SWAP. For example in the
case of $4$-dimensional density matrix
\begin{equation}
X=\left(
\begin{array}{cccc}
 \alpha_{1,1} & \alpha_{1,2} & \alpha_{1,3} & \alpha_{1,4} \\
 \alpha_{2,1} & \alpha_{2,2} & \alpha_{2,3} & \alpha_{2,4} \\
 \alpha_{3,1} & \alpha_{3,2} & \alpha_{3,3} & \alpha_{3,4} \\
 \alpha_{4,1} & \alpha_{4,2} & \alpha_{4,3} & \alpha_{4,4}
\end{array}
\right)
\end{equation}
partial transposition with respect to second subsystems ib obtaines using
\verb+PartialTransposeB[X, 2, 2]+ and it gives
\begin{equation}
X^{T_B}=\left(
\begin{array}{cccc}
 \alpha_{1,1} & \alpha_{2,1} & \alpha_{1,3} & \alpha_{2,3} \\
 \alpha_{1,2} & \alpha_{2,2} & \alpha_{1,4} & \alpha_{2,4} \\
 \alpha_{3,1} & \alpha_{4,1} & \alpha_{3,3} & \alpha_{4,3} \\
 \alpha_{3,2} & \alpha_{4,2} & \alpha_{3,4} & \alpha_{4,4}
\end{array}
\right).
\end{equation}

%%%%%%%%%%%%%%%%%%%%%%%%%%%%%%%%%%%%%%%%%%%%%%%%%%%%%%%%%%%%%%%%%%%%%%%%%%%%%%%%
\subsection{Partial trace}
%%%%%%%%%%%%%%%%%%%%%%%%%%%%%%%%%%%%%%%%%%%%%%%%%%%%%%%%%%%%%%%%%%%%%%%%%%%%%%%%
Operation of tracing out a subsystem can be achived in a similar manner. We 
define tracing map as
\begin{equation}
\Phi_\mathrm{tr}(\rho) = \tr \rho \Id,
\end{equation}
which for gives
\begin{equation}
D_{\Phi_\mathrm{tr}} =
\left(
\begin{array}{cccc}
 1 & 0 & 0 & 1 \\
 0 & 0 & 0 & 0 \\
 0 & 0 & 0 & 0 \\
 1 & 0 & 0 & 1
\end{array}
\right)
\end{equation}
for one qubit. Note that this matrix after reshuffling is equal to $\Id$ and 
thus $\Phi_\tr$ is CP-TP map.
\bibliography{qi}
\bibliographystyle{elsarticle-num}

\end{document}
\endinput
%%
%% End of file `elsarticle-template-num.tex'.
