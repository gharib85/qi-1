
%% This template can be used to write a paper for
%% Computer Physics Communications using LaTeX.
%% For authors who want to write a computer program description,
%% an example Program Summary is included that only has to be
%% completed and which will give the correct layout in the
%% preprint and the journal.
%% The `elsart' style that is used and more information on this style
%% can be found at the Author Gateway (http://authors.elsevier.com) and follow %% the link to "Guide to publishing with Elsevier".
\documentclass{elsart}

%% This list environment is used for the references in the
%% Program Summary
%%
\newcounter{bla}
\newenvironment{refnummer}{%
\list{[\arabic{bla}]}%
{\usecounter{bla}%
 \setlength{\itemindent}{0pt}%
 \setlength{\topsep}{0pt}%
 \setlength{\itemsep}{0pt}%
 \setlength{\labelsep}{2pt}%
 \setlength{\listparindent}{0pt}%
 \settowidth{\labelwidth}{[9]}%
 \setlength{\leftmargin}{\labelwidth}%
 \addtolength{\leftmargin}{\labelsep}%
 \setlength{\rightmargin}{0pt}}}
 {\endlist}

\begin{document}
\begin{frontmatter}

\title{A \LaTeX{} template for CPC Computer Program Descriptions}

\author[a]{First Author\thanksref{author}},
\author[a,b]{Second Author},
\author[b]{Third Author}

\thanks[author]{Corresponding author}

\address[a]{First Address}
\address[b]{Second Address}

\begin{abstract}
  %Type your abstract here.
A submitted program is expected to be of benefit to other physicists or physical chemists, or be an exemplar of good programming practice, or illustrate new or novel programming techniques which are of importance to some branch of computational physics or physical chemistry.

Acceptable program descriptions can take different forms. The following Long Write-Up structure is a suggested structure but it is not obligatory. Actual structure will depend on the length of the program, the extent to which the algorithms or software have already been described in literature, and the detail provided in the user manual.

Your manuscript and figure sources should be submitted through the Elsevier Editorial System (EES) by using the online submission tool at \\ 
http://www.ees.elsevier.com/cpc.

In addition to the manuscript you must supply: the program source code; job control scripts, where applicable; a README file giving the names and a brief description of all the files that make up the package and clear instructions on the installation and execution of the program; sample input and output data for at least one comprehensive test run; and, where appropriate, a user manual. These should be sent, via email as a compressed archive file, to the CPC Program Librarian at cpc@qub.ac.uk.

\begin{flushleft}
  %Insert your suggested PACS number here
PACS: PACS code one; PACS code two; PACS code three; Etc.
\end{flushleft}

\begin{keyword}
Keyword one; Keyword two; Keyword three; Etc.
  % Please give some freely chosen keywords that we can use in a
  % cumulative keyword index.
\end{keyword}

\end{abstract}


\end{frontmatter}

% Computer program descriptions should contain the following
% PROGRAM SUMMARY.

{\bf PROGRAM SUMMARY/NEW VERSION PROGRAM SUMMARY}
  %Delete as appropriate.

\begin{small}
\noindent
{\em Manuscript Title:}                                       \\
{\em Authors:}                                                \\
{\em Program Title:}                                          \\
{\em Journal Reference:}                                      \\
  %Leave blank, supplied by Elsevier.
{\em Catalogue identifier:}                                   \\
  %Leave blank, supplied by Elsevier.
{\em Licensing provisions:}                                   \\
  %enter "none" if CPC non-profit use license is sufficient.
{\em Programming language:}                                   \\
{\em Computer:}                                               \\
  %Computer(s) for which program has been designed.
{\em Operating system:}                                       \\
  %Operating system(s) for which program has been designed.
{\em RAM:} bytes                                              \\
  %RAM in bytes required to execute program with typical data.
{\em Number of processors used:}                              \\
  %If more than one processor.
{\em Supplementary material:}                                 \\
  % Fill in if necessary, otherwise leave out.
{\em Keywords:} Keyword one, Keyword two, Keyword three, etc.  \\
  % Please give some freely chosen keywords that we can use in a
  % cumulative keyword index.
{\em PACS:} PACS code one, PACS code two, PACS code three, etc.                                                  \\
  % see http://www.aip.org/pacs/pacs.html
{\em Classification:}                                         \\
  %Classify using CPC Program Library Subject Index, see (
  % http://cpc.cs.qub.ac.uk/subjectIndex/SUBJECT_index.html)
  %e.g. 4.4 Feynman diagrams, 5 Computer Algebra.
{\em External routines/libraries:}                                      \\
  % Fill in if necessary, otherwise leave out.
{\em Subprograms used:}                                       \\
  %Fill in if necessary, otherwise leave out.
{\em Catalogue identifier of previous version:}*              \\
  %Only required for a New Version summary, otherwise leave out.
{\em Journal reference of previous version:}*                  \\
  %Only required for a New Version summary, otherwise leave out.
{\em Does the new version supersede the previous version?:}*   \\
  %Only required for a New Version summary, otherwise leave out.

{\em Nature of problem:}\\
  %Describe the nature of the problem here.
   \\
{\em Solution method:}\\
  %Describe the method solution here.
   \\
{\em Reasons for the new version:}*\\
  %Only required for a New Version summary, otherwise leave out.
   \\
{\em Summary of revisions:}*\\
  %Only required for a New Version summary, otherwise leave out.
   \\
{\em Restrictions:}\\
  %Describe any restrictions on the complexity of the problem here.
   \\
{\em Unusual features:}\\
  %Describe any unusual features of the program/problem here.
   \\
{\em Additional comments:}\\
  %Provide any additional comments here.
   \\
{\em Running time:}\\
  %Give an indication of the typical running time here.
   \\
{\em References:}
\begin{refnummer}
\item Reference 1         % This is the reference list of the Program Summary
\item Reference 2         % Type references in text as [1], [2], etc.
\item Reference 3         % This list is different from the bibliography, which
                          % you can use in the Long Write-Up.
\end{refnummer}
* Items marked with an asterisk are only required for new versions
of programs previously published in the CPC Program Library.\\
\end{small}

\newpage

% In program descriptions the main text of the paper is listed under
% the heading LONG WRITE-UP.

\hspace{1pc}
{\bf LONG WRITE-UP}

\section{Introduction}

\section{Theoretical background}

\section{Overview of the software structure}

\section{Description of the individual software components}

\section{Installation instructions}

\section{Test run description}

\section{Acknowledgements**}

\section{Appendices**}

\section{References}

** These sections are optional


\end{document}
